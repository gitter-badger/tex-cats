\documentclass[CT4S-EN-RU]{subfiles}

\begin{document}

\section{\caseENGRUS{Other notions in $\Cat$}{ / }{Другие понятия в $\Cat$}}

\begin{blockENG}
In this section we discuss some leftover notions about categories. For example in Section \ref{sec:opposite} we explain a kind of duality for categories, in which arrows are flipped. For example reversing the order in a preorder is an example of this duality, as is the similarity between limits and colimits. In Section \ref{sec:grothendieck construction} we discuss the so-called Grothendieck construction which in some sense graphs functors, and we show that it is useful for transforming databases into the kind of format (RDF) used in scraping data off webpages. We define a general construction for creating categories in Section \ref{sec:comma}. Finally, in Section \ref{sec:arithmetic of categories} we show that precisely the same arithmetic statements that held for sets in Section \ref{sec:arithmetic of sets} hold for categories. 
\end{blockENG}

\begin{blockRUS}
\end{blockRUS}

%%%% Subsection %%%%

\subsection{\caseENGRUS{Opposite categories}{ / }{Дуальные категории}}\label{sec:opposite}\index{functor!contravariant}\index{functor!covariant}

\begin{blockENG}
People used to discuss two different kinds of functors between categories: the so-called {\em covariant functors} and the so-called {\em contravariant functors}. Covariant functors are what we have been calling functors. The reader may have come across the idea of contravariance when considering Exercise \ref{exc:points and opens in Top}.\footnote{Similarly, see Exercise \ref{exc:juris 2}.} There we saw that a continuous mapping of topological spaces $f\taking X\to Y$ does not induce a morphism of orders on their open sets $\Op(X)\to\Op(Y)$; that is not required by the notion of continuity. Instead, a morphism of topological spaces $f\taking X\to Y$ induces a morphism of orders $\Op(Y)\to\Op(X)$, going backwards. So we do not have a functor $\Top\to\PrO$ in this way, but it's quite close. One used to say that $\Op$ is a {\em contravariant functor} $\Top\to\PrO$.
\end{blockENG}

\begin{blockRUS}
\end{blockRUS}

\begin{blockENG}
As important and common as contravariance is, people found that keeping track of which functors were covariant and which were contravariant was a big hassle. Luckily, there is a simple work-around, which simplifies everything: the notion of opposite categories.
\end{blockENG}

\begin{blockRUS}
\end{blockRUS}

\begin{definitionENG}\index{category!opposite}
Let $\mcC$ be a category. The {\em opposite category} of $\mcC$, denoted $\mcC\op$,\index{a symbol!$\mcC\op$} has the same objects as $\mcC$, i.e. $\Ob(\mcC\op)=\Ob(\mcC)$, and for any two objects $c,c'$, one defines
$$\Hom_{\mcC\op}(c,c'):=\Hom_\mcC(c',c).$$
\end{definitionENG}

\begin{definitionRUS}\index{category!opposite}
\end{definitionRUS}

\begin{exampleENG}
If $n\in\NN$ is a natural number and $\ul{n}$ the corresponding discrete category, then $\ul{n}\op=\ul{n}$. Recall  the span category $I=\ul{2}\lcone$ from Definition \ref{def:products in a cat}. Its opposite is the cospan category $I\op=\ul{2}\rcone$, from Definition \ref{def:coproducts in a cat}.
\end{exampleENG}

\begin{exampleRUS}
\end{exampleRUS}

\begin{exerciseENG}
Let $\mcC$ be the category from Example \ref{ex:category of spans}. Draw $\mcC\op$.
\end{exerciseENG}

\begin{exerciseRUS}
\end{exerciseRUS}

\begin{lemmaENG}
Let $\mcC$ and $\mcD$ be categories. One has $(\mcC\op)\op=\mcC$. Also we have $\Fun(\mcC,\mcD)\iso\Fun(\mcC\op,\mcD\op)$. This implies that a functor $\mcC\op\to\mcD$ can be identified with a functor $\mcC\to\mcD\op$.
\end{lemmaENG}

\begin{lemmaRUS}
\end{lemmaRUS}

\begin{proofENG}
This follows straightforwardly from the definitions.
\end{proofENG}

\begin{proofRUS}
\end{proofRUS}

\begin{exerciseENG}
In Exercises \ref{exc:points and opens in Top}, \ref{exc:juris 1}, and \ref{exc:juris 2} there were questions about whether a certain function $\Ob(\mcC)\to\Ob(\mcD)$ extended to a functor $\mcC\to\mcD$. In each case, see if the proposed function would extend to a “contravariant functor” i.e. to a functor $\mcC\op\to\mcD$.
\end{exerciseENG}

\begin{exerciseRUS}
\end{exerciseRUS}

\begin{exampleENG}[Simplicial sets]\label{ex:simplicial set}\index{simplicial set}
Recall from Example \ref{ex:finite linear orders} the category $\bD$ of linear orders $[n]$. For example, $[1]$ is the linear order $0\leq 1$ and $[2]$ is the linear order $0\leq 1\leq2$. Both $[1]$ and $[2]$ are objects of $\bD$.\index{a category!$\bD$} There are 6 morphisms from $[1]$ to $[2]$, which we could denote $$\Hom_{\bD}([1],[2])=\{(0,0), (0,1), (0,2), (1,1), (1,2), (2,2)\}.$$

It may seem strange, but the category $\bD\op$ turns out to be quite useful in algebraic topology. It is the indexing category for a combinatorial approach to the homotopy theory of spaces. That is, we can represent something like the category of spaces and continuous maps using the functor category $\sSet:=\Fun(\bD\op,\Set)$,\index{a category!$\sSet$} which is called the {\em category of simplicial sets}. 

This may seem very complicated compared to something we did earlier, namely simplicial complexes. But simplicial sets have excellent formal properties that simplicial complexes do not. We will not go further with this here, but through the work of Dan Kan, Andr\'{e} Joyal, Jacob Lurie, and many others, simplicial sets have allowed category theory to pierce deeply into the realm of topology and vice versa.
\end{exampleENG}

\begin{exampleRUS}[Simplicial sets]\label{ex:simplicial set}\index{simplicial set}
\end{exampleRUS}

%%%% Subsection %%%%

\subsection{\caseENGRUS{Grothendieck construction}{ / }{Конструкция Гротендика}}\label{sec:grothendieck construction}\index{Grothendieck!construction}

\begin{blockENG}
Let $\mcC$ be a database schema (or category) and let $J\taking\mcC\to\Set$ be an instance. We have been drawing this in table form, but there is another standard way of laying out the data in $J$, called the \href{http://en.wikipedia.org/wiki/Resource_Description_Framework}{\em resource descriptive framework} or RDF. Developed for the web, RDF is a useful format when one does not have a schema in hand, e.g. when scraping information off of a website, one does not know what schema will be best. In these cases, information is stored in so-called RDF triples, which are of the form $$\la\tn{Subject, Predicate, Object}\ra$$\index{RDF}
\end{blockENG}

\begin{blockRUS}
\end{blockRUS}

\begin{blockENG}
For example, one might see something like 
\begin{align}\label{dia:Obama yells at congress}
\begin{tabular}{| lll |}
\bhline
{\bf Subject}&{\bf Predicate}&{\bf Object}\\\bbhline
A01&occurredOn&D13114\\
A01&performedBy&P44\\
A01&actionDescription&Told congress to raise debt ceiling\\
D13114&hasYear&2013\\
D13114&hasMonth&January\\
D13114&hasDay&14\\
P44&FirstName&Barack\\
P44&LastName&Obama\\\bhline
\end{tabular}
\end{align}
\end{blockENG}

\begin{blockRUS}
\end{blockRUS}

\begin{blockENG}
Category-theoretically, it is quite simple to convert a database instance $J\taking\mcC\to\Set$ into an RDF triple store. To do so, we use the {\em Grothendieck construction}, which is more aptly named the category of elements construction, defined below.\footnote{Apparently, Alexander Grothendieck\index{Grothendieck} did not invent this construction, it was discussed prior to Grothendieck's use of it, e.g. by Mac Lane. But more to the point, the term Grothendieck construction is not grammatically suited in the sense that both the following are awkward in English: “the Grothendieck construction of $J$ is ...” (awkward because $J$ is not being constructed but used in a construction) and “the Grothendieck construct for $J$ is...” (awkward because it just is). The term {\em category of elements} is more descriptive and easier to use grammatically.}
\end{blockENG}

\begin{blockRUS}
\end{blockRUS}

\begin{definitionENG}\label{def:grothendieck}\index{category!of elements}
Let $\mcC$ be a category and let $J\taking\mcC\to\Set$ be a functor. The {\em category of elements of $J$}, denoted $\int_\mcC J$,\index{a symbol!$\int$} is defined as follows:
\begin{align*}
\Ob(\varint_\mcC J):=&\;\;\{(C,x)\|C\in\Ob(\mcC), x\in J(C)\}.\\
\Hom_{\int_\mcC J}((C,x),(C',x')):=&\;\;\{f\taking C\to C'\|J(f)(x)=x'\}.
\end{align*}

There is a natural functor $\pi_J\taking\int_\mcC J\too\mcC$. It sends each object $(C,x)\in\Ob(\int_\mcC J)$ to the object $C\in\Ob(\mcC)$. And it sends each morphism $f\taking (C,x)\to (C',x')$ to the morphism $f\taking C\to C'$. We call $\pi_J$ the {\em projection functor}.
\end{definitionENG}

\begin{definitionRUS}\label{def:grothendieck}\index{category!of elements}
\end{definitionRUS}

\begin{exampleENG}
Let $A$ be a set, and consider it as a discrete category. We saw in Exercise \ref{exc:indexed sets as functors} that a functor $S\taking A\to\Set$ is the same thing as an $A$-indexed set, as discussed in Section \ref{sec:indexed sets}. We will follow Definition \ref{def:indexed sets} and for each $a\in A$ write $S_a:=S(a)$.

What is the category of elements of a functor $S\taking A\to\Set$? The objects of $\int_AS$ are pairs $(a,s)$ where $a\in A$ and $s\in S(a)$. Since $A$ has nothing but identity morphisms, $\int_AS$ has nothing but identity morphisms; i.e. it is the discrete category on a set. In fact that set is the disjoint union $$\varint_AS=\bigsqcup_{a\in A}S_a.$$ The functor $\pi_S\taking\int_AS\to A$ sends each element in $S_a$ to the element $a\in A$. 

One can see this as a kind of histogram. For example, let $A=\{{\tt BOS, NYC, LA, DC}\}$ and let $S\taking A\to\Set$ assign 
\begin{align*}
S_{{\tt BOS}}&=\{{\tt Abby, Bob, Casandra}\},\\
S_{\tt NYC}&=\emptyset,\\
S_{\tt LA}&=\{{\tt John, Jim}\}, \tn{and}\\
S_{\tt DC}&=\{{\tt Abby,Carla}\}.
\end{align*}
Then the category of elements of $S$ would look like the (discrete) category at the top: 
\begin{align}\label{dia:elements for cities}
\varint_AS=\parbox{2.7in}{\fbox{\xymatrix@=10pt{
\LTO{(BOS,Abby)}\\
\LTO{(BOS,Bob)}&\hspace{.3in}&\LTO{(LA,John)}&\LTO{(DC,Abby)}\\
\LTO{(BOS,Casandra)}&&\LTO{(LA,Jim)}&\LTO{(DC,Carla)}
}}}
\end{align}
$$
\hsp\xymatrix{~\ar[d]_{\pi_S}\\~}
$$
$$
\;\;A=\fbox{\xymatrix@=33pt{\hspace{.25in}\LTO{BOS}&\LTO{NYC}&\LTO{LA}&\LTO{DC}\hspace{.2in}}}
$$

We also see that the category of elements construction has converted an $A$-indexed set into a relative set over $A$, as in Definition \ref{def:relative sets}.
\end{exampleENG}

\begin{exampleRUS}
\end{exampleRUS}

\begin{blockENG}
The above example does not show at all how the Grothendieck construction transforms a database instance into an RDF triple store. The reason is that our database schema was $A$, a discrete category that specifies no connections between data (it simply collects the data into bins). 
So lets examine a more interesting database schema and instance. This is taken from \cite{Sp2}.
\end{blockENG}

\begin{blockRUS}
\end{blockRUS}

\begin{applicationENG}\index{RDF!as category of elements}
Consider the schema below, which we first encountered in Example \ref{ex:department store 3}:
\begin{align}\label{dia:basic cat}\mcC:=\MainCatLarge{}\end{align}
And consider the instance $J\taking\mcC\to\Set$, which we first encountered in (\ref{dia:instance on maincat}) and (\ref{dia:instance on maincat 2})

\begin{align*}
&\footnotesize
\begin{tabular}{| l || l | l | l | l |}\bhline
\multicolumn{5}{| c |}{{\tt Employee}}\\\bhline 
{\bf ID}&{\bf first}&{\bf last}&{\bf manager}&{\bf worksIn}\\\bbhline 101&David&Hilbert&103&q10\\\hline 102&Bertrand&Russell&102&x02\\\hline 103&Emmy&Noether&103&q10\\\bhline
\end{tabular}&\hsp\footnotesize
\begin{tabular}{| l || l | l |}\bhline
\multicolumn{3}{| c |}{{\tt Department}}\\
\bhline {\bf ID}&{\bf name}&{\bf secretary}\\\bbhline q10&Sales&101\\\hline x02&Production&102\\\bhline
\end{tabular}
\end{align*}\vspace{.1in}

\begin{align*}\footnotesize
\begin{tabular}{| l ||}\bhline
\multicolumn{1}{| c |}{{\tt FirstNameString}}\\\bhline
{\bf ID}\\\bbhline Alan\\\hline Bertrand\\\hline Carl\\\hline David\\\hline Emmy\\\bhline
\end{tabular}\hspace{.6in}\footnotesize
\begin{tabular}{| l ||}\bhline
\multicolumn{1}{| c |}{{\tt LastNameString}}\\\bhline
{\bf ID}\\\bbhline Arden\\\hline Hilbert\\\hline Jones\\\hline Noether\\\hline Russell\\\bhline
\end{tabular}\hspace{.6in}\footnotesize
\begin{tabular}{| l ||}\bhline
\multicolumn{1}{| c |}{{\tt DepartmentNameString}}\\\bhline
{\bf ID}\\\bbhline Marketing\\\hline Production\\\hline Sales\\\bhline
\end{tabular}
\end{align*}  

The category of elements of $J\taking\mcC\to\Set$ looks like this:

\begin{align}\label{dia:Grothendieck}
\hspace{0in}\varint_\mcC J=\parbox{3.9in}{
\fbox{
\xymatrix@=.1pt{
&\LTO{101}\ar@/_2.5pc/[ddddl]+<-3pt,3pt>_{\tin{first}}\ar@/_1.5pc/[ddrrr]+<-4pt,0pt>^{\tin{last}}\ar@/_1pc/[rr]+<-3pt,-3pt>_-{\tin{manager}}\ar@/^1.8pc/[rrrrrr]^{\tin{worksIn}}&\LTO{\;\;102}&\LTO{103}&&&&\LTO{q10}&\LTO{x02}\ar@/^1.6pc/[llllll]+<4pt,-2pt>_{\tin{secretary}}\ar@/^1.2pc/[lddd]+<6pt,2pt>^(.4){\tin{name}}\\\parbox{.1in}{~\\\vspace{.6in}~}\\\LTO{Alan}&&&&\LTO{Hilbert}&\hspace{.4in}&\hspace{.4in}&\LTO{Production}\\\LTO{\;\;Bertrand}&&&&\LTO{Russell}&&&\LTO{\hspace{-.1in}Sales}\\\LTO{\hspace{.2in}David}&&&&\LTO{Noether}&&&\LTO{Marketing}\\\LTO{Emmy}&&&&\LTO{Arden}\\\LTO{Carl}&&&&\LTO{Jones}}}\\
\xymatrix{\hspace{1.6in}&\ar[d]^{\pi_J}\\&~}}\\
\nonumber \mcC=\parbox{3.9in}{\hspace{.1in}\fbox{
			\xymatrix@=9pt{&\LTO{Employee}\ar@<.5ex>[rrrrr]^{\tn{worksIn}}\ar@(l,u)[]+<5pt,10pt>^{\tn{manager}}\ar[dddl]_{\tn{first}}\ar[dddr]^{\tn{last}}&&&\hspace{0in}&&\LTO{Department}\ar@<.5ex>[lllll]^{\tn{secretary}}\ar[ddd]^{\tn{name}}\\\\\\\LTO{\parbox{.3in}{\tt \scriptsize FirstNameString}}&&\LTO{LastNameString}&~&~&~&\LTO{DepartmentNameString}
			}}}
\end{align}~\\

In the above drawing (\ref{dia:Grothendieck}) of $\int_\mcC J$, we left out 10 arrows for ease of readability, for example, we left out an arrow $\LTO{102}\Too{\tt first}\LTO{Bertrand}$.

For the punchline, how do we see the category of elements $\int_\mcC J$ as an RDF triple store? For each arrow in $\int_\mcC J$, we take the triple consisting of the source vertex, the arrow name, and the target vertex. So our triple store would include triples such as $\la{\tt 102\;\; first\;\; Bertrand}\ra$ and $\la{\tt 101\;\; manager\;\; 103}\ra$.
\end{applicationENG}

\begin{applicationRUS}\index{RDF!as category of elements}
\end{applicationRUS}

\begin{exerciseENG}
Come up with a schema and instance whose category of elements contains (at least) the data from (\ref{dia:Obama yells at congress}).
\end{exerciseENG}

\begin{exerciseRUS}
\end{exerciseRUS}

\begin{sloganENG}
The Grothendieck construction takes structured, boxed-up data and flattens it by throwing it all into one big space. The projection functor is then tasked with remembering which box each datum originally came from.
\end{sloganENG}

\begin{sloganRUS}
\end{sloganRUS}

\begin{exerciseENG}\label{exc:FSM as elements of monoid action}\index{finite state machine}
Recall from Section \ref{sec:FSMs} that a finite state machine is a free monoid $(\List(\Sigma),[\;],\plpl)$ acting on a set $X$. Recall also that we can consider a monoid as a category $\mcM$ with one object and a monoid action as a set-valued functor $F\taking\mcM\to\Set$, (see Section \ref{sec:monoids as cats}). In the case of Figure \ref{fig:fsa} the monoid in question is $\List(a,b)$, which can be drawn as the schema
$$\fbox{\xymatrix{\monOb\ar@(ul,dl)[]_(.3)a\ar@(ul,dl)[]_(.7){~}\ar@(ur,dr)[]^(.3)b\ar@(ur,dr)[]^(.7){~}}}$$
and the functor $F\taking\mcM\to\Set$ is recorded in an action table in Example \ref{ex:action table}. What is $\int_\mcM F$? How does it relate to the picture in Figure \ref{fig:fsa}?
\end{exerciseENG}

\begin{exerciseRUS}\label{exc:FSM as elements of monoid action}\index{finite state machine}
\end{exerciseRUS}

%%%% Subsection %%%%

\subsection{\caseENGRUS{Full subcategory}{ / }{Полная подкатегория}}\index{subcategory!full}

\begin{definitionENG}\label{def:full subcategory}
Let $\mcC$ be a category and let $X\ss\Ob(\mcC)$ be a set of objects in $\mcC$. The {\em full subcategory of $\mcC$ spanned by $X$} is the category, which we denote by $\mcC_{\Ob=X}$, with objects $\Ob(\mcC_{\Ob=X}):=X$ and with morphisms $\Hom_{\mcC_{\Ob=X}}(x,x'):=\Hom_\mcC(x,x')$.
\end{definitionENG}

\begin{definitionRUS}\label{def:full subcategory}
\end{definitionRUS}

\begin{exampleENG}
The following are examples of full subcategories. We will name them in the form “$X$ inside of $Y$”, and each time we mean that $X$ and $Y$ are names of categories, the category $X$ can be considered as a subcategory of the category $Y$ in some sense, and it is full. In other words, all morphisms in $Y$ “count” as morphisms in $X$.
\begin{itemize}
\item Finite sets inside of sets, $\Fin\ss\Set$;
\item Finite sets of the form $\ul{n}$ inside of $\Fin$;
\item Linear orders of the form $[n]$ inside of all finite linear orders, $\bD\ss\FLin$;
\item Groups inside of monoids, $\Grp\ss\Mon$;
\item Monoids inside of categories, $\Mon\ss\Cat$;
\item Sets inside of graphs, $\Set\ss\Grph$;
\item Partial orders (resp. linear orders) inside of $\PrO$;
\item Discrete categories (resp. indiscrete categories) inside of $\Cat$;
\end{itemize}
\end{exampleENG}

\begin{exampleRUS}
\end{exampleRUS}

\begin{remarkENG}
A subcategory $\mcC\ss\mcD$ is (up to isomorphism) just a functor $i\taking\mcC\to\mcD$ that happens to be injective on objects and arrows. The subcategory is full if and only if $i$ is a full functor in the sense of Definition \ref{def:full faithful}.
\end{remarkENG}

\begin{remarkRUS}
\end{remarkRUS}

\begin{exampleENG}
Let $\mcC$ be a category, let $X\ss\Ob(\mcC)$ be a set of objects, and let $\mcC_{\Ob=X}$ denote the full subcategory of $\mcC$ spanned by $X$. We can realize this as a fiber product of categories. Indeed, recall that for any set, we can form the indiscrete category on that set; see Example \ref{ex:indiscrete cat equiv to terminal}. In fact, we have a functor $Ind\taking\Set\to\Cat$.\index{a functor!$Ind\taking\Set\to\Cat$} Thus our function $X\to\Ob(\mcC)$ can be converted into a functor between indiscrete categories $Ind(X)\to Ind(\Ob(\mcC))$. There is also a functor $\mcC\to Ind(\Ob(\mcC))$ sending each object to itself. Then the full subcategory of $\mcC$ spanned by $X$ is the fiber product of categories,
$$\xymatrix{\mcC_{\Ob=X}\ar[r]\ar[d]&\mcC\ar[d]\\Ind(X)\ar[r]&Ind(\Ob(\mcC))}$$
\end{exampleENG}

\begin{exampleRUS}
\end{exampleRUS}

\begin{exerciseENG}
Including all identities and all compositions, how many morphisms are there in the full subcategory of $\Set$ spanned by the objects $\{\ul{0},\ul{1},\ul{2}\}$? Write them out.
\end{exerciseENG}

\begin{exerciseRUS}
\end{exerciseRUS}

%%%% Subsection %%%%

\subsection{\caseENGRUS{Comma categories}{ / }{Категории запятой}}\label{sec:comma}\index{category!comma}

\begin{blockENG}
Category theory includes a highly developed and interoperable catalogue of materials and production techniques. One such is the comma category.
\end{blockENG}

\begin{blockRUS}
\end{blockRUS}

\begin{definitionENG}\label{def:comma category}
Let $\mcA,\mcB,$ and $\mcC$ be categories and let $F\taking\mcA\to\mcC$ and $G\taking\mcB\to\mcC$ be functors. The {\em comma category of $\mcC$ morphisms from $F$ to $G$}, denoted $(F\down_\mcC G)$ or simply $(F\down G)$,\index{a symbol!$(F\down G)$} is the category with objects $$\Ob(F\down G)=\{(a,b,f)\|a\in\Ob(\mcA), b\in\Ob(\mcB), f\taking F(a)\to G(b)\tn{ in }\mcC\}$$ and for any two objects $(a,b,f)$ and $(a',b',f')$ the set $\Hom_{(F\down G)}((a,b,f),(a',b',f'))$ of morphisms $(a,b,f)\too(a',b',f')$ is 
$$\{(q,r)\|q\taking a\to a'\tn{ in }\mcA,\;\; r\taking b\to b'\tn{ in }\mcB,\tn{ such that } f'\circ F(q)=G(r)\circ f\}.$$
In pictures,
$$\Hom_{(F\down G)}((a,b,f),(a',b',f')):=\left\{\parbox{2in}{\xymatrix{
a\ar[d]_q&F(a)\ar@{}[dr]|{\checkmark}\ar[r]^f\ar[d]_{F(q)}&G(b)\ar[d]^{G(r)}&b\ar[d]^r\\
a'&F(a')\ar[r]_{f'}&G(b')&b'
}}\right\}$$
We refer to the diagram $\mcA\To{F}\mcC\From{G}\mcB$ (in $\Cat$) as the {\em setup} for the comma category $(F\down G)$.

There is a canonical functor $(F\down G)\to\mcA$ called {\em left projecton}, sending $(a,b,f)$ to $a$, and a canonical functor $(F\down G)\to\mcB$ called {\em right projection}, sending $(a,b,f)$ to $b$. 
\end{definitionENG}

\begin{definitionRUS}\label{def:comma category}
\end{definitionRUS}

\begin{blockENG}
A setup $\mcA\To{F}\mcC\From{G}\mcB$ is reversable; i.e. we can flip it to obtain $\mcB\To{G}\mcC\From{F}\mcA$. However, note that $(F\down G)$ is different than (i.e. almost never equivalent to) $(G\down F)$, unless every arrow in $\mcC$ is an isomorphism.
\end{blockENG}

\begin{blockRUS}
\end{blockRUS}

\begin{sloganENG}
When two categories $\mcA,\mcB$ can be interpreted in a common setting $\mcC$, the comma category integrates them by recording how to move from $\mcA$ to $\mcB$ inside $\mcC$.
\end{sloganENG}

\begin{sloganRUS}
\end{sloganRUS}

\begin{exampleENG}
Let $\mcC$ be a category and $I\taking\mcC\to\Set$ a functor. In this example we show that the comma category construction captures the notion of taking the category of elements $\int_\mcC I$; see Definition \ref{def:grothendieck}. 

Consider the set $\ul{1}$, the category $Disc(\ul{1})$, and the functor $F\taking Disc(\ul{1})\to\Set$ sending the unique object to the set $\ul{1}$. We use the comma category setup $\ul{1}\Too{F}\Set\Fromm{I}\mcC$. There is an isomorphism of categories 
$$\int_\mcC I\iso (F\down I).$$
Indeed, an object in $(F\down I)$ is a triple $(a,b,f)$ where $a\in\Ob(\ul{1}), b\in\Ob(\mcC)$, and $f\taking F(a)\to I(b)$ is a morphism in $\Set$. There is only one object in $\ul{1}$, so this reduces to a pair $(b,f)$ where $b\in\Ob(\mcC)$ and $f\taking \singleton\to I(b)$. The set of functions $\singleton\to I(b)$ is isomorphic to $I(b)$, as we saw in Exercise \ref{exc:generator for set}. So we have reduced $\Ob(F\down I)$ to the set of pairs $(b,x)$ where $b\in\Ob(\mcC)$ and $x\in I(b)$; this is $\Ob(\int_\mcC I)$. Because there is only one function $\ul{1}\to\ul{1}$, a morphism $(b,x)\to(b',x')$ in $(F\down I)$ boils down to a morphism $r\taking b\to b'$ such that the diagram 
$$\xymatrix{\ul{1}\ar[r]^x\ar@{=}[d]&I(b)\ar[d]^{I(r)}\\\ul{1}\ar[r]_{x'}&I(b')}$$
commutes. But such diagrams are in one-to-one correspondence with the diagrams needed for morphisms in $\int_\mcC I$.
\end{exampleENG}

\begin{exampleRUS}
\end{exampleRUS}

\begin{exerciseENG}
Let $\mcC$ be a category and let $c,c'\in\Ob(\mcC)$ be objects. Consider them as functors $c,c'\taking\ul{1}\to\mcC$, and consider the setup $\ul{1}\Too{c}\mcC\Fromm{c'}\ul{1}$. What is the comma category $(c\down c')$?
\end{exerciseENG}

\begin{exerciseRUS}
\end{exerciseRUS}

%%%% Subsection %%%%

\subsection{\caseENGRUS{Arithmetic of categories}{ / }{Арифметика категорий}}\label{sec:arithmetic of categories}

\begin{blockENG}
In Section \ref{sec:arithmetic of sets}, we summarized some of the properties of products, coproducts, and exponentials for sets, attempting to show that they lined up precisely with familiar arithmetic properties of natural numbers. Astoundingly, we can do the same for categories.
\end{blockENG}

\begin{blockRUS}
\end{blockRUS}

\begin{blockENG}
In the following proposition, we denote the coproduct of two categories $\mcA$ and $\mcB$ by the notation $\mcA+\mcB$ rather than $\mcA\sqcup\mcB$. We also denote the functor category $\Fun(\mcA,\mcB)$ by $\mcB^\mcA$. Finally, we use $\ul{0}$ and $\ul{1}$ to refer to the discrete category on 0 and on 1 object, respectively.
\end{blockENG}

\begin{blockRUS}
\end{blockRUS}

\begin{propositionENG}\label{prop:arithmetic of cats}\index{category!arithmetic of}
The following isomorphisms exist for any small categories $\mcA,\mcB,$ and $\mcC$.

\begin{itemize}
\item $\mcA+\ul{0}\iso \mcA$
\item $\mcA + \mcB\iso \mcB + \mcA$
\item $(\mcA + \mcB) + \mcC \iso \mcA + (\mcB + \mcC)$
\item $\mcA\times\ul{0}\iso\ul{0}$
\item $\mcA\times\ul{1}\iso \mcA$
\item $\mcA\times \mcB\iso \mcB\times \mcA$
\item $(\mcA\times\mcB)\times\mcC\iso\mcA\times(\mcB\times\mcC)$
\item $\mcA\times(\mcB+\mcC)\iso (\mcA\times \mcB)+(\mcA\times \mcC)$
\item $\mcA^{\ul{0}}\iso \ul{1}$
\item $\mcA^{\ul{1}}\iso \mcA$
\item $\ul{0}^\mcA\iso\ul{0}$,\;\; if $\mcA\neq\ul{0}$
\item $\ul{1}^\mcA\iso\ul{1}$
\item $\mcA^{\mcB+\mcC}\iso \mcA^\mcB\times \mcA^\mcC$
\item $(\mcA^\mcB)^\mcC\iso \mcA^{\mcB\times \mcC}$
\end{itemize}
\end{propositionENG}

\begin{propositionRUS}\label{prop:arithmetic of cats}\index{category!arithmetic of}
\end{propositionRUS}

\begin{proofENG}
These are standard results; see \cite{Mac}.
\end{proofENG}

\begin{proofRUS}
\end{proofRUS}

\end{document}
