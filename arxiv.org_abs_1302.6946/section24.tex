\documentclass[CT4S-EN-RU]{subfiles}

\begin{document}

\section{\caseENGRUS{Products and coproducts}{ / }{Произведения и копроизведения}}\label{sec:prods and coprods in set}

\begin{blockENG}
In this section we introduce two concepts that are likely to be familiar, although perhaps not by their category-theoretic names, product and coproduct. Each is an example of a large class of ideas that exist far beyond the realm of sets.
\end{blockENG}

\begin{blockRUS}
В данном разделе мы введем два понятия, которые, скорее всего, известны читателю, хотя и, возможно, не под их теоретико-категорными именами: произведение и копроизведение. Каждое из них является представителем большого класса идей, выходящих далеко за рамки теории множеств. 
\end{blockRUS}

%%%% Subsection %%%%

\subsection{\caseENGRUS{Products}{ / }{Произведения}}\label{sec:products}

\begin{definitionENG}
Let $X$ and $Y$ be sets. The {\em product of $X$ and $Y$}\index{products!of sets}, denoted $X\times Y$,\index{a symbol!$\times$} is defined as the set of ordered pairs $(x,y)$ where $x\in X$ and $y\in Y$. Symbolically, $$X\times Y=\{(x,y)\|x\in X,\;\; y\in Y\}.$$ There are two natural {\em projection functions} $\pi_1\taking X\times Y\to X$ and $\pi_2\taking X\times Y\to Y$.\index{projection functions}\index{product!projection functions}
$$\xymatrix@=15pt{&X\times Y\ar[ddr]^{\pi_2}\ar[ddl]_{\pi_1}\\\\X&&Y}$$
\end{definitionENG}

\begin{definitionRUS}
Пусть $X$ и $Y$ — множества. {\em Произведение $X$ и $Y$}\index{произведения!множеств}, обозначаемое $X\times Y$,\index{символ!$\times$} определяется как множество упорядоченных пар $(x,y)$, где $x\in X$ и $y\in Y$. На языке символов: $$X\times Y=\{(x,y)\|x\in X,\;\; y\in Y\}.$$ Имеются две естественные {\em функции проектирования} (или просто {\em проекции}) $\pi_1\taking X\times Y\to X$ и $\pi_2\taking X\times Y\to Y$.\index{функции проектирования}\index{произведение!функции проектирования}
$$\xymatrix@=15pt{&X\times Y\ar[ddr]^{\pi_2}\ar[ddl]_{\pi_1}\\\\X&&Y}$$
\end{definitionRUS}

\begin{exampleENG}[Grid of dots]\label{ex:grid1}\index{product!as grid}
Let $X=\{1,2,3,4,5,6\}$ and $Y=\{\clubsuit,\diamondsuit,\heartsuit,\spadesuit\}$. Then we can draw $X\times Y$ as a 6-by-4 grid of dots, and the projections as projections
\begin{align}
\parbox{2.9in}{\begin{center}\small $X\times Y$\vspace{-.1in}\end{center}\fbox{
\xymatrix@=10pt{
\LMO{(1,\clubsuit)}&\LMO{(2,\clubsuit)}&\LMO{(3,\clubsuit)}&\LMO{(4,\clubsuit)}&\LMO{(5,\clubsuit)}&\LMO{(6,\clubsuit)}\\
\LMO{(1,\diamondsuit)}&\LMO{(2,\diamondsuit)}&\LMO{(3,\diamondsuit)}&\LMO{(4,\diamondsuit)}&\LMO{(5,\diamondsuit)}&\LMO{(6,\diamondsuit)}\\
\LMO{(1,\heartsuit)}&\LMO{(2,\heartsuit)}&\LMO{(3,\heartsuit)}&\LMO{(4,\heartsuit)}&\LMO{(5,\heartsuit)}&\LMO{(6,\heartsuit)}\\
\LMO{(1,\spadesuit)}&\LMO{(2,\spadesuit)}&\LMO{(3,\spadesuit)}&\LMO{(4,\spadesuit)}&\LMO{(5,\spadesuit)}&\LMO{(6,\spadesuit)}\\
}}}
\parbox{.9in}{
\xymatrix{~\ar[rr]^{\pi_2}&&~}
}
\parbox{.3in}{\begin{center}\small $Y$\vspace{-.1in}\end{center}\fbox{
\xymatrix@=10pt{
\LMO{\clubsuit}\\\LMO{\diamondsuit}\\\LMO{\heartsuit}\\\LMO{\spadesuit}
}}}
\\\nonumber
\parbox{1in}{\hspace{-1.95in}\xymatrix{~\ar[dd]_{\pi_1}\\\\~}}
\\\nonumber
\parbox{2.9in}{\hspace{-1.2in}\fbox{
\xymatrix@=24pt{
\LMO{1}&\LMO{2}&\LMO{3}&\LMO{4}&\LMO{5}&\LMO{6}
}}\begin{center}\hspace{-2.6in}\small$X$\end{center}}
\end{align}
\end{exampleENG}

\begin{exampleRUS}[Сетка из точек]\label{ex:grid1}\index{произведение!как сетка}
Пусть $X=\{1,2,3,4,5,6\}$ и $Y=\{\clubsuit,\diamondsuit,\heartsuit,\spadesuit\}$. Тогда мы можем изобразить $X\times Y$ в виде состоящей из точек таблицы 6-на-4, а функции проектирования — в виде следующих геометрических проекций
\begin{align}
\parbox{2.9in}{\begin{center}\small $X\times Y$\vspace{-.1in}\end{center}\fbox{
\xymatrix@=10pt{
\LMO{(1,\clubsuit)}&\LMO{(2,\clubsuit)}&\LMO{(3,\clubsuit)}&\LMO{(4,\clubsuit)}&\LMO{(5,\clubsuit)}&\LMO{(6,\clubsuit)}\\
\LMO{(1,\diamondsuit)}&\LMO{(2,\diamondsuit)}&\LMO{(3,\diamondsuit)}&\LMO{(4,\diamondsuit)}&\LMO{(5,\diamondsuit)}&\LMO{(6,\diamondsuit)}\\
\LMO{(1,\heartsuit)}&\LMO{(2,\heartsuit)}&\LMO{(3,\heartsuit)}&\LMO{(4,\heartsuit)}&\LMO{(5,\heartsuit)}&\LMO{(6,\heartsuit)}\\
\LMO{(1,\spadesuit)}&\LMO{(2,\spadesuit)}&\LMO{(3,\spadesuit)}&\LMO{(4,\spadesuit)}&\LMO{(5,\spadesuit)}&\LMO{(6,\spadesuit)}\\
}}}
\parbox{.9in}{
\xymatrix{~\ar[rr]^{\pi_2}&&~}
}
\parbox{.3in}{\begin{center}\small $Y$\vspace{-.1in}\end{center}\fbox{
\xymatrix@=10pt{
\LMO{\clubsuit}\\\LMO{\diamondsuit}\\\LMO{\heartsuit}\\\LMO{\spadesuit}
}}}
\\\nonumber
\parbox{1in}{\hspace{-1.95in}\xymatrix{~\ar[dd]_{\pi_1}\\\\~}}
\\\nonumber
\parbox{2.9in}{\hspace{-1.2in}\fbox{
\xymatrix@=24pt{
\LMO{1}&\LMO{2}&\LMO{3}&\LMO{4}&\LMO{5}&\LMO{6}
}}\begin{center}\hspace{-2.6in}\small$X$\end{center}}
\end{align}
\end{exampleRUS}

\begin{applicationENG}
A traditional (Mendelian) way to predict the genotype of offspring based on the genotype of its parents is by the use of \href{http://en.wikipedia.org/wiki/Punnett_square}{Punnett squares}. If $F$ is the set of possible genotypes for the female parent and $M$ is the set of possible genotypes of the male parent, then $F\times M$ is drawn as a square, called a Punnett square, in which every combination is drawn. 
\end{applicationENG}

\begin{applicationRUS}
Традиционный (по Менделю) способ предсказания генотипа потомков основываясь на генотипе родителей использует \href{https://ru.wikipedia.org/wiki/%D0%A0%D0%B5%D1%88%D1%91%D1%82%D0%BA%D0%B0_%D0%9F%D0%B5%D0%BD%D0%BD%D0%B5%D1%82%D0%B0}{решётку Пеннета}. Если $F$ это множество возможных генотипов женского родителя и $M$ это множество возможных генотипов мужского родителя, то $F\times M$ изображается в виде квадрата, называемого решеткой Пеннета, в которой показана каждая комбинация. 
\end{applicationRUS}

\begin{exerciseENG}
How many elements does the set $\{a,b,c,d\}\times\{1,2,3\}$ have?
\end{exerciseENG}

\begin{exerciseRUS}
Сколько всего элементов в множестве $\{a,b,c,d\}\times\{1,2,3\}$?
\end{exerciseRUS}

\begin{applicationENG}
Suppose we are conducting experiments about the mechanical properties of materials, as in Application~\ref{app:force-extension}. For each material sample we will produce multiple data points in the set $\fakebox{extension}\times\fakebox{force}\iso\RR\times\RR$.
\end{applicationENG}

\begin{applicationRUS}
Предположим, мы проводим эксперименты, касающиеся механических свойств материалов, как в Приложении~\ref{app:force-extension}. Для каждого образца материала мы получим несколько точек данных в множестве $\fakebox{extension}\times\fakebox{force}\iso\RR\times\RR$.
\end{applicationRUS}

\begin{remarkENG}
It is possible to take the product of more than two sets as well. For example, if $A,B,$ and $C$ are sets then $A\times B\times C$ is the set of triples, 
$$A\times B\times C:=\{(a,b,c)\|a\in A, b\in B, c\in C\}.$$

This kind of generality is useful in understanding multiple dimensions, e.g. what physicists mean by 10-dimensional space. It comes under the heading of {\em limits}, which we will see in Section~\ref{sec:lims and colims in a cat}.
\end{remarkENG}

\begin{remarkRUS}
Можно также брать произведение более чем двух множеств. Например, если $A,B$ и $C$ это множества, то $A\times B\times C$ это множество троек, 
$$A\times B\times C:=\{(a,b,c)\|a\in A, b\in B, c\in C\}.$$

Обобщение такого рода полезно для понимания многомерности, в частности того, что физики называют 10-мерным пространством. И все это охватывается понятием {\em пределов}, которое мы рассмотрим в Разделе~\ref{sec:lims and colims in a cat}.
\end{remarkRUS}

\begin{exampleENG}\label{ex:R2}
Let $\RR$\index{a symbol!$\RR$} be the set of real numbers. By $\RR^2$ we mean $\RR\times\RR$ (though see Exercise~\ref{exc:two R2s}). Similarly, for any $n\in\NN$, we define $\RR^n$ to be the product of $n$ copies of $\RR$. 

According to \cite{Pen}, Aristotle seems to have conceived of space as something like $S:=\RR^3$ and of time as something like $T:=\RR$. Spacetime, had he conceived of it, would probably have been $S\times T\iso\RR^4$. He of course did not have access to this kind of abstraction, which was probably due to Descartes. 
\end{exampleENG}

\begin{exampleRUS}\label{ex:R2}
Пусть $\RR$\index{символ!$\RR$} означает множество действительных чисел. Под $\RR^2$ мы имеем в виду $\RR\times\RR$ (см. Упражнение~\ref{exc:two R2s}). Аналогично, для любого $n\in\NN$ определим $\RR^n$ как произведение $n$ копий $\RR$ [т.е. множество конечных последовательностей действительных чисел длины $n$]. 

Согласно \cite{Pen}, Аристотель, похоже, воспринимал пространство, как нечто вроде $S:=\RR^3$, и время, как $T:=\RR$. Пространством-временем, если бы он рассматривал его, было бы вероятно $S\times T\iso\RR^4$. На самом деле, конечно, такого рода абстракции были для него недоступны, и восходят скорее к Декарту. 
\end{exampleRUS}

\begin{exerciseENG}
Let $\ZZ$ denote the set of integers, and let $+\taking\ZZ\times\ZZ\to\ZZ$ denote the addition function and $\cdot\taking\ZZ\times\ZZ\to\ZZ$ denote the multiplication function. Which of the following diagrams commute?
\sexc $$\xymatrix{
\ZZ\times\ZZ\times\ZZ\ar[rr]^-{(a,b,c)\mapsto(a\cdot b,a\cdot c)}\ar[d]_{(a,b,c)\mapsto(a+b,c)}&\hsp&\ZZ\times\ZZ\ar[d]^{(x,y)\mapsto x+y}\\
\ZZ\times\ZZ\ar[rr]_{(x,y)\mapsto xy}&&\ZZ}
$$
\item $$
\xymatrix{
\ZZ\ar[rr]^{x\mapsto (x,0)}\ar[drr]_{\id_\ZZ}&&\ZZ\times\ZZ\ar[d]^{(a,b)\mapsto a\cdot b}\\&&\ZZ}
$$
\item$$
\xymatrix{
\ZZ\ar[rr]^{x\mapsto (x,1)}\ar[drr]_{\id_\ZZ}&&\ZZ\times\ZZ\ar[d]^{(a,b)\mapsto a\cdot b}\\&&\ZZ}
$$
\endsexc
\end{exerciseENG}

\begin{exerciseRUS}
Пусть $\ZZ$ обозначает множество целых чисел, и пусть $+\taking\ZZ\times\ZZ\to\ZZ$ — функцию сложения, и $\cdot\taking\ZZ\times\ZZ\to\ZZ$ — функцию умножения. Какие из следующих диаграмм коммутируют?
\sexc $$\xymatrix{
\ZZ\times\ZZ\times\ZZ\ar[rr]^-{(a,b,c)\mapsto(a\cdot b,a\cdot c)}\ar[d]_{(a,b,c)\mapsto(a+b,c)}&\hsp&\ZZ\times\ZZ\ar[d]^{(x,y)\mapsto x+y}\\
\ZZ\times\ZZ\ar[rr]_{(x,y)\mapsto xy}&&\ZZ}
$$
\item $$
\xymatrix{
\ZZ\ar[rr]^{x\mapsto (x,0)}\ar[drr]_{\id_\ZZ}&&\ZZ\times\ZZ\ar[d]^{(a,b)\mapsto a\cdot b}\\&&\ZZ}
$$
\item$$
\xymatrix{
\ZZ\ar[rr]^{x\mapsto (x,1)}\ar[drr]_{\id_\ZZ}&&\ZZ\times\ZZ\ar[d]^{(a,b)\mapsto a\cdot b}\\&&\ZZ}
$$
\endsexc
\end{exerciseRUS}

%% Subsubsection %%

\subsubsection{\caseENGRUS{Universal property for products}{ / }{Универсальное свойство произведений}}

\begin{lemmaENG}[Universal property for product]\label{lemma:up for prod}\index{universal property!products}
Let $X$ and $Y$ be sets. For any set $A$ and functions $f\taking A\to X$ and $g\taking A\to Y$, there exists a unique function $A\to X\times Y$ such that the following diagram commutes%
\footnote{The symbol $\forall$ is read “for all”; the symbol $\exists$ is read “there exists”, and the symbol $\exists!$ is read “there exists a unique”. So this diagram is intended to express the idea that for any functions $f\taking A\to X$ and $g\taking A\to Y$, there exists a unique function $A\to X\times Y$ for which the two triangles commute.}
\begin{align}\label{dia:univ prop for products}
\xymatrix@=15pt{&X\times Y\ar[ldd]_{\pi_1}\ar[rdd]^{\pi_2}\\\\X\ar@{}[r]|{\checkmark}&&Y\ar@{}[l]|{\checkmark}\\\\&A\ar[luu]^{\forall f}\ar[ruu]_{\forall g}\ar@{-->}[uuuu]^{\exists !}}
\end{align}
We might write the unique function as $$\prodmap{f}{g}\taking A\to X\times Y.$$
\end{lemmaENG}

\begin{lemmaRUS}[Универсальное свойство произведения]\label{lemma:up for prod}\index{произведения!универсальное свойство}
Пусть $X$ и $Y$ — множества. Для любого множества $A$ и функций $f\taking A\to X$ и $g\taking A\to Y$ существует единственная функция $A\to X\times Y$, [зависящая от $A, f, g$ и] такая, что следующая диаграмма коммутирует:%
\footnote{Символ $\forall$ читается «для всех»; символ $\exists$ — «существует», а символ $\exists!$ — «существует единственный». Таким образом, эта диаграмма должна выражать идею, что для любых функций $f\taking A\to X$ и $g\taking A\to Y$, существует единственная функция $A\to X\times Y$, для которой оба треугольника коммутируют.}
\begin{align}\label{dia:univ prop for products}
\xymatrix@=15pt{&X\times Y\ar[ldd]_{\pi_1}\ar[rdd]^{\pi_2}\\\\X\ar@{}[r]|{\checkmark}&&Y\ar@{}[l]|{\checkmark}\\\\&A\ar[luu]^{\forall f}\ar[ruu]_{\forall g}\ar@{-->}[uuuu]^{\exists !}}
\end{align}
Мы могли бы обозначить эту единственную функцию как $$\prodmap{f}{g}\taking A\to X\times Y.$$
\end{lemmaRUS}

\begin{proofENG}
Suppose given $f,g$ as above. To provide a function $\ell\taking A\to X\times Y$ is equivalent to providing an element $\ell(a)\in X\times Y$ for each $a\in A$. We need such a function for which $\pi_1\circ \ell=f$ and $\pi_2\circ \ell=g$. An element of $X\times Y$ is an ordered pair $(x,y)$, and we can use $\ell(a)=(x,y)$ if and only if $x=\pi_1(x,y)=f(a)$ and $y=\pi_2(x,y)=g(a)$. So it is necessary and sufficient to define $$\prodmap{f}{g}(a):=(f(a),g(a))$$ for all $a\in A$.
\end{proofENG}

\begin{proofRUS}
Предположим, даны такие $f,g$, какие требуются выше. Указать функцию $\ell\taking A\to X\times Y$ эквивалентно тому, чтобы указать элемент $\ell(a)\in X\times Y$ для каждого $a\in A$. Нам нужна такая функция, что $\pi_1\circ \ell=f$ и $\pi_2\circ \ell=g$. Произвольный элемент $X\times Y$ — это упорядоченная пара $(x,y)$, поэтому мы получаем, что для $\ell(a)=(x,y)$ должны выполняться равенства $x=\pi_1(x,y)=f(a)$ и $y=\pi_2(x,y)=g(a)$. Таким образом, необходимо и достаточно определить $$\prodmap{f}{g}(a):=(f(a),g(a))$$ для всех $a\in A$.
\end{proofRUS}

\begin{exampleENG}[Grid of dots, continued]\label{ex:grid2}
We need to see the universal property of products as completely intuitive. Recall that if $X$ and $Y$ are sets, say of cardinalities $|X|=m$ and $|Y|=n$ respectively, then $X\times Y$ is an $m\times n$ grid of dots, and it comes with two canonical projections $X\From{\pi_1}X\times Y\To{\pi_2}Y$. These allow us to extract from every grid element $z\in X\times Y$ its column $\pi_1(z)\in X$ and its row $\pi_2(z)\in Y$.

Suppose that each person in a classroom picks an element of $X$ and an element of $Y$. Thus we have functions $f\taking C\to X$ and $g\taking C\to Y$. But isn't picking a column and a row the same thing as picking an element in the grid? The two functions $f$ and $g$ induce a unique function $C\to X\times Y$. And how does this function $C\to X\times Y$ compare with the original functions $f$ and $g$? The commutative diagram (\ref{dia:univ prop for products}) sums up the obvious connection. 
\end{exampleENG}

\begin{exampleRUS}[Сетки из точек, продолжение]\label{ex:grid2}
В универсальном свойстве произведений следует видеть нечто совершенно интуитивное. Напомним, что если $X$ и $Y$ — множества, имеющие мощности, скажем, $|X|=m$ и $|Y|=n$ соответственно, то $X\times Y$ — это сетка из точек размером $m\times n$, сопровождаемая двумя каноническими проекциями $X\From{\pi_1}X\times Y\To{\pi_2}Y$. Они позволяют нам узнать по каждому элементу $z\in X\times Y$ его столбец $\pi_1(z)\in X$ и строку $\pi_2(z)\in Y$.

Предположим, что каждый человек в зале выберет произвольный элемент $X$ (столбец) и элемент $Y$ (строку). Тогда мы получим функции $f\taking C\to X$ и $g\taking C\to Y$. Но не будет ли выбор столбца и строки одновременно тем же самым, что и выбор элемента сетки? То есть, две функции $f$ и $g$ порождают единственную функцию $C\to X\times Y$. И как же эта функция $C\to X\times Y$ связана с исходными $f$ and $g$? Коммутативная диаграмма (\ref{dia:univ prop for products}) выражает эту очевидную связь. 
\end{exampleRUS}

\begin{exampleENG}
Let $\RR$ be the set of real numbers. The origin in $\RR$ is an element of $\RR$. As you showed in Exercise~\ref{exc:generator for set}, we can view this (or any) element of $\RR$ as a function $z\taking\singleton\to\RR$, where $\singleton$ is any set with one element. Our function $z$ “picks out the origin”. Thus we can draw functions 
$$\xymatrix@=15pt{&\singleton\ar[ddr]^z\ar[ddl]_z\\\\\RR&&\RR}
$$
The universal property for products guarantees a function $\singleton\to\RR\times\RR$, which will be the origin in $\RR^2.$
\end{exampleENG}

\begin{exampleRUS}
Пусть $\RR$ это множество действительных чисел. Начало системы отсчета в $\RR$ это просто элемент $\RR$. Как вы уже показали в Упражнении~\ref{exc:generator for set}, мы можем смотреть на такие элементы $\RR$ как на функции $z\taking\singleton\to\RR$, где $\singleton$ — любое множество с одним элементом. Наша функция $z$ «выбирает начало системы отсчета». Таким образом, мы можем изобразить функции 
$$\xymatrix@=15pt{&\singleton\ar[ddr]^z\ar[ddl]_z\\\\\RR&&\RR}
$$
После этого универсальное свойство произведений позволяет нам получить функцию $\singleton\to\RR\times\RR$, которая будет началом системы отсчета в $\RR^2.$ 
\end{exampleRUS}

\begin{remarkENG}
Given sets $X, Y,$ and $A$, and functions $f\taking A\to X$ and $g\taking A\to Y$, there is a unique function $A\to X\times Y$ that commutes with $f$ and $g$. We call it {\em the induced function $A\to X\times Y$},\index{induced function} meaning the one that arises in light of $f$ and $g$.
\end{remarkENG}

\begin{remarkRUS}
Для данных множеств $X, Y$ и $A$, а также функций $f\taking A\to X$ и $g\taking A\to Y$, имеется единственная функция $A\to X\times Y$, коммутирующая с $f$ and $g$. Мы называем ее {\em индуцированной функцией}\index{индуцированная функция} $A\to X\times Y$, имея ввиду ту, что возникает из $f$ и $g$. 
\end{remarkRUS}

\begin{exerciseENG}
For every set $A$ there is some nice relationship between the following three sets: $$\Hom_{\Set}(A,X), \hsp \Hom_\Set(A,Y), \hsp \text{and} \hsp\Hom_\Set(A,X\times Y).$$ What is it?

Hint: Do not be alarmed: this problem is a bit “recursive” in that you'll use products in your formula.
\end{exerciseENG}

\begin{exerciseRUS}
Для каждого множества $A$ имеется некое замечательное отношение между следующими тремя множествами: $$\Hom_{\Set}(A,X), \hsp \Hom_\Set(A,Y), \hsp \text{и} \hsp\Hom_\Set(A,X\times Y).$$ Какое?

Подсказка: Без паники: эта задача немного «рекурсивна» по части использования произведения в формулах. 
\end{exerciseRUS}

\begin{exerciseENG}~
\sexc Let $X$ and $Y$ be sets. Construct the “swap map” $s\taking X\times Y\to Y\times X$ using only the universal property for products. If $\pi_1\taking X\times Y\to X$ and $\pi_2\taking X\times Y\to Y$ are the projection functions, write $s$ in terms of the symbols $\qtE{\pi_1}, \qtE{\pi_2}, \qtE{(\ ,\ )},$ and $\qtE{\circ}$. 
\item Can you prove that $s$ is a isomorphism using only the universal property for product?
\endsexc
\end{exerciseENG}

\begin{exerciseRUS}~
\sexc Пусть $X$ и $Y$ — множества. Постройте «отображение перестановки» $s\taking X\times Y\to Y\times X$ используя только универсальное свойство произведений. Если $\pi_1\taking X\times Y\to X$ и $\pi_2\taking X\times Y\to Y$ — функции проектирования, запишите $s$ через символы $\qtR{\pi_1}, \qtR{\pi_2}, \qtR{\prodmap{-}{-}}$ и $\qtR{- \circ -}$. 
\item Можете ли вы доказать, что $s$ — изоморфизм, используя только универсальное свойство произведения?
\endsexc
\end{exerciseRUS}

\begin{exampleENG}\label{ex:product to product}
Suppose given sets $X,X', Y, Y'$ and functions $m\taking X\to X'$ and $n\taking Y\to Y'$. We can use the universal property of products to construct a function $s\taking X\times Y\to X'\times Y'$.  Here's how.

The universal property (Lemma~\ref{lemma:up for prod}) says that to get a function from any set $A$ to $X'\times Y'$, we need two functions, namely some $f\taking A\to X'$ and some $g\taking A\to Y'$. Here $A=X\times Y$. 

What we have readily available are the two projections $\pi_1\taking X\times Y\to X$ and $\pi_2\taking X\times Y\to Y$. But we also have $m\taking X\to X'$ and $n\taking Y\to Y'$. Composing, we set $f:=m\circ \pi_1$ and $g:=n\circ\pi_2$.
$$\xymatrix{
&X'\times Y'\ar[dl]_{\pi_1'}\ar[dr]^{\pi_2'}\\
X'&&Y'\\
X\ar[u]^m&&Y\ar[u]_n\\
&X\times Y\ar[ul]^{\pi_1}\ar[ur]_{\pi_2}\ar@{-->}[uuu]
}
$$
The dotted arrow is often called the {\em product} of $m\taking X\to X'$ and $n\taking Y\to Y'$ and is denoted simply by 
$$m\times n\taking X\times Y\to X'\times Y'.$$
\end{exampleENG}

\begin{exampleRUS}\label{ex:product to product}
Предположим, даны множества $X, X', Y, Y'$ и функции $m\taking X\to X'$ и $n\taking Y\to Y'$. Мы можем использовать универсальное свойство произведений, чтобы построить функцию $s\taking X\times Y\to X'\times Y'$ следующим образом.

Универсальное свойство (Лемма~\ref{lemma:up for prod}) говорит, что для определения функции из любого множества $A$ в $X'\times Y'$ нам требуются две функции, а именно $f\taking A\to X'$ и $g\taking A\to Y'$. Здесь $A=X\times Y$. 

У нас уже есть две проекции $\pi_1\taking X\times Y\to X$ и $\pi_2\taking X\times Y\to Y$. Также у нас имеются $m\taking X\to X'$ и $n\taking Y\to Y'$. Образуя композицию функций, положим $f:=m\circ \pi_1$ и $g:=n\circ\pi_2$.
$$\xymatrix{
&X'\times Y'\ar[dl]_{\pi_1'}\ar[dr]^{\pi_2'}\\
X'&&Y'\\
X\ar[u]^m&&Y\ar[u]_n\\
&X\times Y\ar[ul]^{\pi_1}\ar[ur]_{\pi_2}\ar@{-->}[uuu]
}
$$
Пунктирную стрелку часто называют {\em произведением} $m\taking X\to X'$ и $n\taking Y\to Y'$ и обозначают просто  
$$m\times n\taking X\times Y\to X'\times Y'.$$ 
\end{exampleRUS}

%% Subsubsection %%

\subsubsection{\caseENGRUS{Ologging products}{ / }{Произведения в ологах}}\label{sec:ologging products}

\begin{blockENG}
Given two objects $c,d$ in an olog, there is a canonical label $\qtE{c\times d}$ for their product $c\times d$, written in terms of the labels $\qtE{c}$ and $\qtE{d}$. Namely, $$\qtE{c\times d}:=\tn{a pair }(x,y)\tn{ where }x\tn{ is }\qtE{c}\tn{ and }y\tn{ is }\qtE{d}.$$ The projections $c\from c\times d\to d$ can be labeled “yields, as $x$,” and “yields, as $y$,” respectively.
\end{blockENG}

\begin{blockRUS}
Для данных двух объектов олога $c,d$, имеется каноническая метка $\qtR{c\times d}$ их произведения $c\times d$, сформулированная в терминах меток $\qtR{c}$ и $\qtR{d}$. А именно, $$\qtR{c\times d}:=\tn{пара }(x,y),\tn{ где }x\tn{ это }\qtR{c}\tn{ и }y\tn{ это }\qtR{d}.$$ Проекции $c\from c\times d\to d$ могут быть помечены «выдает в качестве $x$,» и “выдает в качестве $y$,” соответственно. 
\end{blockRUS}

\begin{blockENG}
Suppose that $e$ is another object and $p\taking e\to c$ and $q\taking e\to d$ are two arrows. By the universal property of products (Lemma~\ref{lemma:up for prod}), $p$ and $q$ induce a unique arrow $e\to c\times d$ making the evident diagrams commute. This arrow can be labeled
\begin{center}
yields, insofar as it $\qtR{p}\;\qtR{c}$ and $\qtR{q}\;\qtR{d}$, 
\end{center}
\end{blockENG}

\begin{blockRUS}
Предположим, $e$ — другой объект, а $p\taking e\to c$ и $q\taking e\to d$ две стрелки. Вследствие универсального свойства произведений (Лемма~\ref{lemma:up for prod}), $p$ и $q$ порождают единственную стрелку $e\to c\times d$, делающую очевидные диаграммы коммутативными. Эта стрелка может быть помечена
\begin{center}
выдает $\qtR{c}\;\qtR{p}$ и $\qtR{d}\;\qtR{q}$, 
\end{center}
\end{blockRUS}

\begin{exampleENG}
Every car owner owns at least one car, but there is no obvious function $\fakebox{a car owner}\to\fakebox{a car}$ because he or she may own more than one. One good choice would be the car that the person drives most often, which we'll call his or her primary car. Also, given a person and a car, an economist could ask how much utility the person would get out of the car. From all this we can put together the following olog involving products:
$$
\xymatrixnocompile{\obox{O}{.7in}{a car owner}\LAL{dd}{is}\ar[ddrr]_(.35){\parbox{.45in}{\scriptsize owns, as primary,}}\LA{rr}{\parbox{.8in}{\rr\scriptsize yields, insofar as it is a person and owns, as primary, a car,}}&\ar@{}[d]^(.4){\checkmark}&
\obox{P\times C}{1in}{a pair $(x,y)$ where $x$ is a person and $y$ is a car}\ar@/^1pc/[ddll]^(.7){\tn{yields, as }x,}\LA{dd}{\tn{yields, as }y,}\LA{rr}{\parbox{.7in}{\scriptsize has as associated utility}}&&\obox{V}{.8in}{a dollar value}\\&&\\\obox{P}{.5in}{a person}&&\obox{C}{.4in}{a car}
}
$$
\end{exampleENG}

\begin{exampleRUS}
Каждый владелец автомобиля владеет по крайней мере одним автомобилем, однако нет очевидной функции $\fakebox{a car owner}\to\fakebox{a car}$, поскольку он (или она) могут владеть более чем одной машиной. Подходящим выбором может оказаться автомобиль, которым этот человек пользуется наиболее часто, и который мы назовем основной его (или ее) машиной. Кроме того, о каждом человеке и автомобиле экономист мог бы спросить: сколько выгоды данный человек может получить от данной машины? Из всего этого мы можем собрать следующий олог, включающий произведения:
$$
\xymatrixnocompile{\obox{O}{.7in}{владелец автомобиля}\LAL{dd}{является}\ar[ddrr]_(.35){\parbox{.45in}{\scriptsize владеет основным}}\LA{rr}{\parbox{.8in}{\rr\scriptsize выдает себя в качестве человека и основную машину, которой владеет}}&\ar@{}[d]^(.4){\checkmark}&
\obox{P\times C}{1in}{пара $(x,y)$, где $x$ это человек и $y$ это машина}\ar@/^1pc/[ddll]^(.7){\tn{выдает в качестве }x}\LA{dd}{\tn{выдает в качестве }y}\LA{rr}{\parbox{.7in}{\scriptsize дает выгоду}}&&\obox{V}{.8in}{денежная сумма}\\&&\\\obox{P}{.5in}{человек}&&\obox{C}{.4in}{автомобиль}
}
$$
\end{exampleRUS}

%%%% Subsection %%%%

\subsection{\caseENGRUS{Coproducts}{ / }{Копроизведения}}\label{sec:coproducts}

\begin{definitionENG}\label{def:coproduct}
Let $X$ and $Y$ be sets. The {\em coproduct of $X$ and $Y$}\index{coproducts!of sets}, denoted $X\sqcup Y$,\index{a symbol!$\sqcup$} is defined as the “disjoint union” of $X$ and $Y$, i.e. the set for which an element is either an element of $X$ or an element of $Y$. If something is an element of both $X$ and $Y$ then we include both copies, and distinguish between them, in $X\sqcup Y$. See Example~\ref{ex:coproduct}

There are two natural inclusion functions $i_1\taking X\to X\sqcup Y$ and $i_2\taking Y\to X\sqcup Y$.\index{inclusion functions}\index{coproduct!inclusion functions}
$$\xymatrix@=15pt{X\ar[ddr]_{i_1}&&Y\ar[ddl]^{i_2}\\\\&X\sqcup Y}$$
\end{definitionENG}

\begin{definitionRUS}\label{def:coproduct}
Пусть $X$ и $Y$ — множества. {\em Копроизведение $X$ и $Y$}\index{копроизведение!множеств}, обозначаемое $X\sqcup Y$,\index{символ!$\sqcup$} определяется как «несвязное объединение» $X$ и $Y$, то есть множество, каждым элементом которого является либо элемент $X$, либо элемент $Y$. В случае, когда некий элемент принадлежит одновременно $X$ и $Y$, мы вкладываем в $X\sqcup Y$ две его копии и считаем их различными. См. Пример~\ref{ex:coproduct}

Имеются две естественные {\em функции вложения} $i_1\taking X\to X\sqcup Y$ и $i_2\taking Y\to X\sqcup Y$.\index{функции вложения}\index{копроизведение!функции вложения}
$$\xymatrix@=15pt{X\ar[ddr]_{i_1}&&Y\ar[ddl]^{i_2}\\\\&X\sqcup Y}$$
\end{definitionRUS}

\begin{exampleENG}\label{ex:coproduct}
The coproduct of $X:=\{a,b,c,d\}$ and $Y:=\{1,2,3\}$ is $$X\sqcup Y\iso\{a,b,c,d,1,2,3\}.$$ The coproduct of $X$ and itself is $$X\sqcup X\iso\{i_1a,i_1b,i_1c,i_1d,i_2a,i_2b,i_2c,i_2d\}$$ 
The names of the elements in $X\sqcup Y$ are not so important. What's important are the inclusion maps $i_1,i_2$, which ensure that we know where each element of $X\sqcup Y$ came from.
\end{exampleENG}

\begin{exampleRUS}\label{ex:coproduct}
Копроизведение $X:=\{a,b,c,d\}$ и $Y:=\{1,2,3\}$ — это $$X\sqcup Y\iso\{a,b,c,d,1,2,3\}.$$ Копроизведение $X$ с самим собой — это $$X\sqcup X\iso\{i_1a,i_1b,i_1c,i_1d,i_2a,i_2b,i_2c,i_2d\}$$ 
Имена элементов в $X\sqcup Y$ не так важны. Важны отображения вложения $i_1,i_2$, которые удостоверяют, что мы знаем, откуда возникает каждый элемент $X\sqcup X$.
\end{exampleRUS}

\begin{exampleENG}[Airplane seats]\label{ex:airplanes}
\begin{align}\label{dia:airplane}
\xymatrix@=15pt{
\obox{X}{.8in}{an economy-class seat in an airplane}\LAL{ddr}{is}&&\obox{Y}{.7in}{a first-class seat in an airplane}\LA{ddl}{is}\\\\
&\obox{X\sqcup Y}{.7in}{a seat in an airplane}
}
\end{align}
\end{exampleENG}

\begin{exampleRUS}[Места в самолете]\label{ex:airplanes}
\begin{align}\label{dia:airplane}
\xymatrix@=15pt{
\obox{X}{.8in}{место эконом-класса в самолете}\LAL{ddr}{является}&&\obox{Y}{.7in}{место первого-класса в самолете}\LA{ddl}{является}\\\\
&\obox{X\sqcup Y}{.7in}{место в самолете}
}
\end{align}
\end{exampleRUS}

\begin{exerciseENG}
Would you say that \fakebox{a phone} is the coproduct of \fakebox{a cellphone} and \fakebox{a landline phone}? 
\end{exerciseENG}

\begin{exerciseRUS}
Сказали бы вы, что \fakebox{телефон} является копроизведением \fakebox{мобильного телефона} and \fakebox{стационарного телефона}?%
\endnote{
TODO (дать наброски решений некоторых упражнений)
}
\end{exerciseRUS}

\begin{exampleENG}[Disjoint union of dots]\label{ex:coprod of dots}
\begin{align}
\parbox{2.4in}{\begin{center}\small $X\sqcup Y$\vspace{-.1in}\end{center}\fbox{
\xymatrix@=15pt{
\LMO{\clubsuit}&\LMO{1}&\LMO{2}&\LMO{3}&\LMO{4}&\LMO{5}&\LMO{6}\\\LMO{\diamondsuit}\\\LMO{\heartsuit}\\\LMO{\spadesuit}
}}}
\parbox{.9in}{
\xymatrix{~&&\ar[ll]_{i_2}~}
}
\parbox{.3in}{\begin{center}\small $Y$\vspace{-.1in}\end{center}\fbox{
\xymatrix@=15pt{
\LMO{\clubsuit}\\\LMO{\diamondsuit}\\\LMO{\heartsuit}\\\LMO{\spadesuit}
}}}
\\\nonumber
\parbox{1in}{\hspace{-1.4in}\xymatrix{~\\\\\ar[uu]_{i_1}}}
\\\nonumber
\parbox{2.1in}{\hspace{-1.3in}\fbox{
\xymatrix@=15pt{
\LMO{1}&\LMO{2}&\LMO{3}&\LMO{4}&\LMO{5}&\LMO{6}
}}\begin{center}\hspace{-2.6in}\small$X$\end{center}}
\end{align}
\end{exampleENG}

\begin{exampleRUS}[Несвязное объединение точек]\label{ex:coprod of dots}
\begin{align}
\parbox{2.4in}{\begin{center}\small $X\sqcup Y$\vspace{-.1in}\end{center}\fbox{
\xymatrix@=15pt{
\LMO{\clubsuit}&\LMO{1}&\LMO{2}&\LMO{3}&\LMO{4}&\LMO{5}&\LMO{6}\\\LMO{\diamondsuit}\\\LMO{\heartsuit}\\\LMO{\spadesuit}
}}}
\parbox{.9in}{
\xymatrix{~&&\ar[ll]_{i_2}~}
}
\parbox{.3in}{\begin{center}\small $Y$\vspace{-.1in}\end{center}\fbox{
\xymatrix@=15pt{
\LMO{\clubsuit}\\\LMO{\diamondsuit}\\\LMO{\heartsuit}\\\LMO{\spadesuit}
}}}
\\\nonumber
\parbox{1in}{\hspace{-1.4in}\xymatrix{~\\\\\ar[uu]_{i_1}}}
\\\nonumber
\parbox{2.1in}{\hspace{-1.3in}\fbox{
\xymatrix@=15pt{
\LMO{1}&\LMO{2}&\LMO{3}&\LMO{4}&\LMO{5}&\LMO{6}
}}\begin{center}\hspace{-2.6in}\small$X$\end{center}}
\end{align}
\end{exampleRUS}

%% Subsubsection %%

\subsubsection{\caseENGRUS{Universal property for coproducts}{ / }{Универсальное свойство копроизведений}}

\begin{lemmaENG}[Universal property for coproduct]\label{lemma:up for coprod}
Let $X$ and $Y$ be sets. For any set $A$ and functions $f\taking X\to A$ and $g\taking Y\to A$, there exists a unique function $X\sqcup Y\to A$ such that the following diagram commutes\index{coproducts!universal property of}
$$
\xymatrix@=15pt{&A\\\\X\ar[uur]^{\forall f}\ar[ddr]_{i_1}&&Y\ar[uul]_{\forall g}\ar[ddl]^{i_2}\\\\&X\sqcup Y\ar@{-->}[uuuu]^{\exists!}}
$$
We might write the unique function as%
\footnote{We are about to use a two-line symbol, which is a bit unusual. In what follows a certain function $X\sqcup Y\to A$ is being denoted by the symbol $\coprodmap{f}{g}$.}
$$\coprodmap{f}{g}\taking X\sqcup Y\to A.$$
\end{lemmaENG}

\begin{lemmaRUS}[Универсальное свойство копроизведения]\label{lemma:up for coprod}
Пусть $X$ и $Y$ — множества. Для любого множества $A$ и функций $f\taking X\to A$ и $g\taking Y\to A$ имеется единственная функция $X\sqcup Y\to A$, [зависящая от $A, f, g$ и] такая, что следующая диаграмма коммутирует\index{копроизведения!универсальное свойство}:
$$
\xymatrix@=15pt{&A\\\\X\ar[uur]^{\forall f}\ar[ddr]_{i_1}&&Y\ar[uul]_{\forall g}\ar[ddl]^{i_2}\\\\&X\sqcup Y\ar@{-->}[uuuu]^{\exists!}}
$$
Мы можем обозначить эту единственную функцию как%
\footnote{Мы намерены использовать двухстрочный символ, что немного необычно. Ниже отдельная функция $X\sqcup Y\to A$ обозначается символом $\coprodmap{f}{g}$.}
$$\coprodmap{f}{g}\taking X\sqcup Y\to A.$$
\end{lemmaRUS}

\begin{proofENG}
Suppose given $f,g$ as above. To provide a function $\ell\taking X\sqcup Y\to A$ is equivalent to providing an element $f(m)\in A$ is for each $m\in X\sqcup Y$. We need such a function such that $\ell\circ i_1=f$ and $\ell\circ i_2=g$. But each element $m\in X\sqcup Y$ is either of the form $i_1x$ or $i_2y$, and cannot be of both forms. So we assign 
$$\coprodmap{f}{g}(m)=\begin{cases}f(x)&\tn{if } m=i_1x,\\ g(y) &\tn{if }m=i_2y.\end{cases}$$
This assignment is necessary and sufficient to make all relevant diagrams commute.
\end{proofENG}

\begin{proofRUS}
Предположим, даны такие $f,g$, как описано выше. Задать функцию $\ell\taking X\sqcup Y\to A$ эквивалентно тому, чтобы задать элемент $f(m)\in A$ для каждого $m\in X\sqcup Y$. Нам нужна такая функция, что $\ell\circ i_1=f$ и $\ell\circ i_2=g$. Но каждый элемент $m\in X\sqcup Y$ имеет вид либо $i_1x$, либо $i_2y$, и не может быть одновременно ими обоими. Поэтому мы положим 
$$\coprodmap{f}{g}(m):=\begin{cases}f(x)&\tn{если } m=i_1x,\\ g(y) &\tn{если }m=i_2y.\end{cases}$$
Это равенство необходимо и достаточно, чтобы сделать соответствующие диаграммы коммутирующими.
\end{proofRUS}

\begin{exampleENG}[Airplane seats, continued]
The universal property of coproducts says the following. Any time we have a function $X\to A$ and a function $Y\to A$, we get a unique function $X\sqcup Y\to A$. For example, every economy class seat in an airplane and every first class seat in an airplane is actually {\em in a particular airplane}. Every economy class seat has a price, as does every first class seat.
\begin{align}
\xymatrix{
&\obox{A}{.9in}{a dollar figure}&\\
\obox{X}{.8in}{an economy-class seat in an airplane}\LA{ru}{has as price}\LA{r}{is}\LAL{dr}{is in}&\obox{X\sqcup Y}{.7in}{a seat in an airplane}\ar@{-->}[d]_{\exists!}\ar@{-->}[u]^{\exists!}\ar@{}[ur]|(.35){\checkmark}\ar@{}[dl]|(.35){\checkmark}\ar@{}[dr]|(.35){\checkmark}\ar@{}[ul]|(.35){\checkmark}&\obox{Y}{.7in}{a first-class seat in an airplane}\LAL{l}{is}\LAL{lu}{has as price}\LA{dl}{is in}\\
&\obox{B}{.7in}{an airplane}&
}
\end{align}
The universal property of coproducts formalizes the following intuitively obvious fact:
\begin{quote}
If we know how economy class seats are priced and we know how first class seats are priced, and if we know that every seat is either economy class or first class, then we automatically know how all seats are priced.
\end{quote}
To say it another way (and using the other induced map):
\begin{quote}
If we keep track of which airplane every economy class seat is in and we keep track of which airplane every first class seat is in, and if we know that every seat is either economy class or first class, then we require no additional tracking for any airplane seat whatsoever.
\end{quote}
\end{exampleENG}

\begin{exampleRUS}[Места в самолете, продолжение]
Универсальное свойство копроизведений говорит о следующем. Каждый раз, когда у нас есть функция $X\to A$ и функция $Y\to A$, у нас есть и функция $X\sqcup Y\to A$. Например, в случае самолетов, каждое место эконом-класса и каждое место первого класса находятся на самом деле {\em в некотором конкретном самолете}. Далее, каждое место эконом-класса имеет цену, также как и каждое место первого класса.
\begin{align}
\xymatrix{
&\obox{A}{.9in}{денежный эквивалент}&\\
\obox{X}{.8in}{место эконом-класса}\LA{ru}{имеет цену}\LA{r}{является}\LAL{dr}{находится в}&\obox{X\sqcup Y}{.7in}{место в самолете}\ar@{-->}[d]_{\exists!}\ar@{-->}[u]^{\exists!}\ar@{}[ur]|(.35){\checkmark}\ar@{}[dl]|(.35){\checkmark}\ar@{}[dr]|(.35){\checkmark}\ar@{}[ul]|(.35){\checkmark}&\obox{Y}{.7in}{место первого класса}\LAL{l}{является}\LAL{lu}{имеет цену}\LA{dl}{находится в}\\
&\obox{B}{.7in}{самолет}&
}
\end{align}
Универсальное свойство копроизведений формализует следующий интуитивно очевидный факт:
\begin{quote}
Если нам известны цены на места эконом-класса и известны цены на места первого класса, и если мы знаем, что каждое место относится либо к эконом-классу, либо к первому классу, то тогда мы автоматически знаем цены на все места.
\end{quote}
Скажем то же другими словами (и используя другое индуцированное отображение):
\begin{quote}
Если мы отслеживаем, в каком самолете находится каждое место эконом-класса, и в каком самолете находится каждое место первого класса, и если мы знаем, что каждое место относится либо к эконом-классу, либо к первому классу, то нам не требуется отслеживать  больше никакие места в самолетах.
\end{quote}
\end{exampleRUS}

\begin{applicationENG}[Piecewise defined curves]
In science, curves are often defined or considered piecewise. For example in testing the mechanical properties of a material, we might be interested in various regions of \href{http://en.wikipedia.org/wiki/Deformation_(engineering)}{deformation}, such as elastic, plastic, or post-fracture. These are three intervals on which the material displays different kinds of properties. 

For real numbers $a<b\in\RR$, let $[a,b]:=\{x\in\RR\|a\leq x\leq b\}$ denote the closed interval. Given a function $[a,b]\to\RR$ and a function $[c,d]\to\RR$, the universal property of coproducts implies that they extend uniquely to a function $[a,b]\sqcup[c,d]\to\RR$, which will appear as a piecewise defined curve.

Often we are given a curve on $[a,b]$ and another on $[b,c]$, where the two curves agree at the point $b$. This situation is described by pushouts, which are mild generalizations of coproducts; see Section~\ref{sec:pushouts}.
\end{applicationENG}

\begin{applicationRUS}[Кусочно-определенные кривые]
В естественных науках кривые зачастую определяются или рассматриваются кусочным образом. Например, при тестировании механических свойств материала, мы можем заинтересоваться отдельными видами \href{https://ru.wikipedia.org/wiki/%D0%94%D0%B5%D1%84%D0%BE%D1%80%D0%BC%D0%B0%D1%86%D0%B8%D1%8F}{деформации}, такими как упругая, пластическая или разрушение. Это три интервала, на которых материал обладает различным поведением. 

Для действительных чисел $a<b\in\RR$, пусть $[a,b]:=\{x\in\RR\|a\leq x\leq b\}$ обозначает замкнутый интервал. Данная функция $[a,b]\to\RR$ и функция $[c,d]\to\RR$ при момощи универсального свойства копроизведений могут быть продолжены совместно до функции $[a,b]\sqcup[c,d]\to\RR$, которая оказывается кусочно-определенной кривой.

Зачастую мы имеем одну кривую на $[a,b]$ и другую на $[b,c]$, причем две кривые совпадают в точке $b$. Эта ситуация описывается выталкивающими квадратами, являющимися небольшим обобщением копроизведений; см. Раздел~\ref{sec:pushouts}.
\end{applicationRUS}

\begin{exerciseENG}\label{exc:coprod}
Write the universal property for coproduct in terms of a relationship between the following three sets: $$\Hom_{\Set}(X,A), \hsp \Hom_\Set(Y,A), \hsp \text{and} \hsp\Hom_\Set(X\sqcup Y,A).$$ 
\end{exerciseENG}

\begin{exerciseRUS}\label{exc:coprod}
Запишите универсальное свойство копроизведений в виде отношения между следующими тремя множествами: $$\Hom_{\Set}(X,A), \hsp \Hom_\Set(Y,A), \hsp \text{и} \hsp\Hom_\Set(X\sqcup Y,A).$$ 
\end{exerciseRUS}

\begin{exampleENG}\label{ex:coproduct1}
In the following olog the types $A$ and $B$ are disjoint, so the coproduct $C=A\sqcup B$ is just the union. $$\fbox{\xymatrix{\smbox{A}{a person}\LA{r}{is}&\smbox{C=A\sqcup B}{a person or a cat}&\smbox{B}{a cat}\LAL{l}{is}}}$$
\end{exampleENG}

\begin{exampleRUS}\label{ex:coproduct1}
В следующем ологе типы $A$ и $B$ не пересекаются, так что копроизведение $C=A\sqcup B$ — это обычное теоретико-множественное объединение. $$\fbox{\xymatrix{\smbox{A}{человек}\LA{r}{является}&\smbox{C=A\sqcup B}{человек или кот}&\smbox{B}{кот}\LAL{l}{является}}}$$
\end{exampleRUS}

\begin{exampleENG}\label{ex:coproduct2}
In the following olog, $A$ and $B$ are not disjoint, so care must be taken to differentiate common elements. $$\fbox{\xymatrixnocompile{\obox{A}{.7in}{\rr an animal that can fly}\LA{rr}{labeled “A” is}&&\obox{C=A\sqcup B}{1.3in}{an animal that can fly (labeled “A”) or an animal that can swim (labeled “B”)}&&\obox{B}{.9in}{\rr an animal that can swim}\LAL{ll}{labeled “B” is}}}$$  Since ducks can both swim and fly, each duck is found twice in $C$, once labeled as a flyer and once labeled as a swimmer.  The types $A$ and $B$ are kept disjoint in $C$, which justifies the name “disjoint union.”
\end{exampleENG}

\begin{exampleRUS}\label{ex:coproduct2}
В следующем ологе $A$ и $B$ пересекаются, так что особое внимание придется уделить различению общих элементов. $$\fbox{\xymatrixnocompile{\obox{A}{.7in}{\rr летающее животное}\LA{rr}{помечено «A»}&&\obox{C=A\sqcup B}{1.3in}{летающее животное (помеченное «A») или водоплавающее животное (помеченное «B»)}&&\obox{B}{.9in}{\rr водоплавающее животное}\LAL{ll}{помечено «B»}}}$$ Поскольку утки одновременно летают и плавают, каждая утка появляется в $C$ дважды, один раз в качестве летуна и один раз в качестве пловца. Типы $A$ и $B$ остаются непересекающимися (не связанными) в $C$, чем объясняется название «несвязное объединение.»
\end{exampleRUS}

\begin{exerciseENG}
Understand Example~\ref{ex:coproduct2} and see if a similar idea would make sense for particles and waves. Make an olog, and choose your wording in accordance with Rules~\ref{rules:types}. How do photons, which exhibit properties of both waves and particles, fit into the coproduct in your olog?
\end{exerciseENG}

\begin{exerciseRUS}
Разберитесь в Примере~\ref{ex:coproduct2} и определите, может ли аналогичная идея быть применена к частицам и волнам. Создайте олог и выберите названия в соответствии с Правилами~\ref{rules:types}. Каким образом фотоны, обладающие одновременно свойствами волн и частиц, попадают в копроизведение в этом ологе?
\end{exerciseRUS}

\begin{exerciseENG}
Following the section above, “Ologging products” page \pageref{sec:ologging products}, come up with a naming system for coproducts, the inclusions, and the universal maps. Try it out by making an olog (involving coproducts) discussing the idea that both a .wav file and a .mp3 file can be played on a modern computer. Be careful that your arrows are valid in the sense of Section~\ref{sec:invalid aspect}.
\end{exerciseENG}

\begin{exerciseRUS}
В соответствии с разделом выше, «Произведения в ологах» с. \pageref{sec:ologging products}, предложите систему обозначений для копроизведений, вложений и универсальных отображений. Испытайте их при создании (влючающего копроизведения) олога, описывающего идею о том, что и .wav, и .mp3 файлы можно проигрывать на современном компьютере. Будьте внимательны в том, чтобы ваши стрелки были корректны в смысле Раздела~\ref{sec:invalid aspect}.
\end{exerciseRUS}

\end{document}
