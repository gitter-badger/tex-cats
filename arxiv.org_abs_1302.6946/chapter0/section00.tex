\documentclass[../main/CT4S-EN-RU]{subfiles}

\begin{document}

\chapter*{\caseENGRUS{Preface}{ / }{Предисловие}}

\begin{blockENG}
An early version of this book was put on line in February 2013 to serve as the textbook for my course \href{http://math.mit.edu/~dspivak/teaching/sp13/}{\text Category Theory for Scientists} taught in the spring semester of 2013 at MIT. During that semester, students provided me with hundreds of comments and questions, which led to a substantial improvement (and the addition of 50 pages) to the original document.
\end{blockENG}

\begin{blockRUS}
Ранняя версия данной книги была представлена в феврале 2013 в качестве учебника для моего курса \href{http://math.mit.edu/~dspivak/teaching/sp13/}{\text Теория категорий для ученых},%
\endnote{
В англоязычной литературе слово sciences обозначает как науки в целом, так и только естественные науки, в этом случае науки гуманитарные называют humanities [см. \href{https://en.wikipedia.org/wiki/The_Two_Cultures}{\text The Two Cultures}]. Выражая робкую надежду, что эта книга пригодится также и представителям последних, мы переводим science/scientist как наука/ученый вообще, не ограничивая их естественнонаучной направленностью.
} который читался в весеннем семестре 2013 в MIT. В течение семестра студенты передали мне сотни комментариев и вопросов, которые привели к существенному улучшению исходного текста (и к добавлению 50 страниц).
\end{blockRUS}

\begin{blockENG}
In the summer of 2013 I signed a contract with the MIT Press to publish a new version of this work under the title {\em Category Theory for the Sciences}. Because I am committed to the open source development model I insisted that a version of this book, namely the one you are reading, remain freely available online. The MIT Press version will of course not be free.
\end{blockENG}

\begin{blockRUS}
Летом 2013 я подписал контракт с MIT Press о публикации новой версии данной работы под заглавием {\em Теория категорий для наук}. Поскольку я являюсь приверженцем подхода открытого исходного кода, я настоял на том, чтобы отдельная версия этой книги, а именно — та, которую вы сейчас читаете, осталась доступной бесплатно онлайн.%
\endnote{
Данная книга не является единственной работой автора в области категорных оснований баз данных и представления знаний в информатике. Желающие увидеть полный список литературы могут обратиться к сайту Категорные Данные \cite{CDSite}, а также к персональной странице Д. И. Спивака \cite{SpHome}.
}
\end{blockRUS}

\begin{blockENG}
Other than the title, there are two main differences between the present version and the MIT Press version. The first difference is that I will do a full edit with the help of professional editors from the Press. The second difference is that I will write up solutions to the book's (approximately 280) exercises; some of these will be included in the published version, whereas the rest will be available by way of a password-protected page, accessible only to professors who teach the subject.
\end{blockENG}

\begin{blockRUS}
Помимо заголовка, есть еще два основных различия данной версии и версии MIT Press. Первое различие состоит в том, что я проведу кардинальное редактирование при помощи профессиональных редакторов. Второе различие — в том, что я напишу решения для упражнений этой книги (приблизительно 280); некоторые из них будут включены в опубликованную версию, тогда как остальные будут размещены на защищенной паролем странице, доступные только преподавателям, ведущим данный предмет. 
\end{blockRUS}

\end{document}
