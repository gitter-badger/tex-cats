% section40.tex

\chapter{\caseENGRUS{Basic category theory}{ / }{Основы теории категорий}}\label{chap:categories}

“{\it ...We know only a very few—and, therefore, very precious—schemes whose unifying powers cross many realms.}” -- Marvin Minsky.\footnote{\cite[Problems of disunity, p. 126]{Min}.}\\\\

Categories, or an equivalent notion, have already been secretly introduced as ologs. One can think of a category as a graph (as in Section \ref{sec:graphs}) in which certain paths have been declared equivalent. (Ologs demand an extra requirement that everything in sight be readable in natural language, and this cannot be part of the mathematical definition of category.) The formal definition of category is given in Definition \ref{def:category}, but it will not be obviously the same as the “graph+path equivalences” notion; the latter was given in Definition \ref{def:schema} as the definition of a {\em schema}. Once we talk about how different categories can be compared using functors (Definition \ref{def:functor}), and how different schemas can be compared using schema mappings (Definition \ref{def:schema morphism}), we will prove that the two notions are equivalent (Theorem \ref{thm:equivalence of categories and schemas}).

