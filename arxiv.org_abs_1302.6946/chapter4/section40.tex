\documentclass[../main/CT4S-EN-RU]{subfiles}

\begin{document}

\section*{\caseENGRUS{Preamble}{ / }{Преамбула}}

\begin{blockENG}
\begin{quote}
\begin{quote}
~\\
“{\it ...We know only a very few—and, therefore, very precious—schemes whose unifying powers cross many realms.}” – Marvin Minsky.\footnote{\cite[Problems of disunity, p. 126]{Min}.}
\end{quote}
\end{quote}
\end{blockENG}

\begin{blockRUS}
\begin{quote}
\begin{quote}
«{\it ...Нам известны очень немногие и поэтому очень ценимые схемы, объединяющая способность которых проникает сквозь многочисленные области.}» – Марвин Минский.\footnote{\cite[Проблемы разъединения, p. 126]{Min}.}~\\~
\end{quote}
\end{quote}
\end{blockRUS}

\begin{blockENG}
Categories, or an equivalent notion, have already been secretly introduced as ologs. One can think of a category as a graph (as in Section~\ref{sec:graphs}) in which certain paths have been declared equivalent. (Ologs demand an extra requirement that everything in sight be readable in natural language, and this cannot be part of the mathematical definition of category.) The formal definition of category is given in Definition~\ref{def:category}, but it will not be obviously the same as the “graph+path equivalences” notion; the latter was given in Definition~\ref{def:schema} as the definition of a {\em schema}. Once we talk about how different categories can be compared using functors (Definition~\ref{def:functor}), and how different schemas can be compared using schema mappings (Definition~\ref{def:schema morphism}), we will prove that the two notions are equivalent (Theorem~\ref{thm:equivalence of categories and schemas}).
\end{blockENG}

\begin{blockRUS}
Категории (или эквивалентное им понятие) уже были тайно введены под именем {\em ологов}. О категории можно думать как о графе (см. Раздел~\ref{sec:graphs}), в котором заданные пути объявлены эквивалентными. (Ологам требуется дополнительное условие читабельности на естественном языке, а это не может стать частью математического определения категории). Формальному определению категории будет посвящено Определение~\ref{def:category}, но это не будет очевидное поняние «граф $+$ эквивалентность путей»; последнее дается в Определении~\ref{def:schema} как определение {\em схемы}. Как только мы обсудим, каким образом различные категории можно сопоставлять при помощи функторов (Определение~\ref{def:functor}), а также, каким образом различные схемы можно сопоставлять при помощи отображений схем (Определение~\ref{def:schema morphism}), мы докажем, что эти два понятия эквивалентны (Теорема~\ref{thm:equivalence of categories and schemas}).
\end{blockRUS}

\end{document}
