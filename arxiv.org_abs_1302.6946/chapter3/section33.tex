\documentclass[../main/CT4S-EN-RU]{subfiles}

\begin{document}

\section{\caseENGRUS{Graphs}{ / }{Графы}}\label{sec:graphs}

\begin{blockENG}
In this course, unless otherwise specified, whenever we speak of graphs we are not talking about curves in the plane, such as parabolas, or pictures of functions generally. We are speaking of systems of vertices and arrows.
\end{blockENG}

\begin{blockRUS}
В данном курсе, если дополнительно не сказано противоположное, когда говорится о графах, не имеются в виду кривые на плоскости, такие как параболы или вообще изображения функций [это уточнение возникло из-за того, что в английском графы и графики называются одним словом; нужно как-то адаптировать]. Мы понимаем под этим термином системы вершин и стрелок.
\end{blockRUS}

\begin{blockENG}
We will take our graphs to be {\em directed}, meaning that every arrow points {\em from} a vertex {\em to} a vertex; rather than merely connecting vertices, arrows will have direction. If $a$ and $b$ are vertices, there can be many arrows from $a$ to $b,$ or none at all. There can be arrows from $a$ to itself. Here is the formal definition in terms of sets and functions.
\end{blockENG}

\begin{blockRUS}
Мы будем считать наши графы {\em ориентированными} [это общепринятый термин из теории графов], имея в виду, что каждая стрелка идет {\em из} вершины {\em в} вершину; другими словами, вместо того, чтобы просто соединять две равноправные вершины, стрелки имеют направление. Если $a$ и $b$ — вершины, у нас может быть много стрелок из $a$ в $b$ или вообще их не быть. Могут быть стрелки из вершины $a$ в себя. Вот формальное определение графов в терминах множеств и функций.
\end{blockRUS}

%%%% Subsection %%%%

\subsection{\caseENGRUS{Definition and examples}{ / }{Определение и примеры}}

\begin{definitionENG}\label{def:graph}\index{graph}
A {\em graph} $G$ consists of a sequence $G:=(V,A,src,tgt)$ where 
\begin{itemize}
\item $V$ is a set, called {\em the set of vertices of $G$} (singular:{\em vertex}),\index{vertex}
\item $A$ is a set, called {\em the set of arrows of $G$},\index{arrow}
\item $src\taking A\to V$ is a function, called {\em the source function for $G$}, and
\item $tgt\taking A\to V$ is a function, called {\em the target function for $G$}.
\end{itemize}
Given an arrow $a\in A$ we refer to $src(a)$ as the {\em source vertex} of $a$ and to $tgt(a)$ as the {\em target vertex} of $a.$
\end{definitionENG}

\begin{definitionRUS}\label{def:graph}\index{graph}
{\em Граф} $G$ задается четверкой $G:=(V,A,src,tgt),$ где
\begin{itemize}
\item $V$ — это множество, называемое {\em множеством вершин $G$},\index{вершина}
\item $A$ — это множество, называемое {\em множеством стрелок $G$},\index{стрелка}
\item $src\taking A\to V$ — это функция, называемая {\em функцией начала стрелок $G$},
\item $tgt\taking A\to V$ — это функция, называемая {\em функцией конца стрелок $G$}.
\end{itemize}
Для данной стрелки $a\in A$ мы называем $src(a)$ ее {\em началом}, а $tgt(a)$ — ее {\em концом}.
\end{definitionRUS}

\begin{blockENG}
To draw a graph, first draw a dot for every element of $V.$ Then for every element $a\in A,$ draw an arrow connecting dot $src(a)$ to dot $tgt(a).$
\end{blockENG}

\begin{blockRUS}
Чтобы изобразить граф, нарисуем сначала точку для каждого элемента $V.$ Затем для каждого элемента $a\in A$ нарисуем стрелку, соединяющую точку $src(a)$ с точкой $tgt(a).$
\end{blockRUS}

\begin{exampleENG}[Graph]\label{ex:graph}
Here is a picture of a graph $G=(V,A,src,tgt)$:
\begin{align}\label{dia:graph}
G:=\parbox{2in}{\fbox{\xymatrix{\bullet^v\ar[r]^f&\bullet^w\ar@/_1pc/[r]_h\ar@/^1pc/[r]^g&\bullet^x\\\bullet^y\ar@(l,u)[]^i\ar@/^1pc/[r]^j&\bullet^z\ar@/^1pc/[l]^k}}}
\end{align} 
We have $V=\{v,w,x,y,z\}$ and $A=\{f,g,h,i,j,k\}.$ The source and target functions $src,tgt\taking A\to V$ can be captured in the table to the left below:
\begin{align*}
\begin{array}{| l || l | l |}\bhline
{\bf A}&{\bf src}&{\bf tgt}\\\bbhline
f&v&w\\\hline
g&w&x\\\hline
h&w&x\\\hline
i&y&y\\\hline
j&y&z\\\hline
k&z&y\\\bhline
\end{array}
\hspace{1in}
\begin{array}{| l |}\bhline
{\bf V}\\\bbhline
v\\\hline
w\\\hline
x\\\hline
y\\\hline
z\\\bhline
\end{array}
\end{align*}
In fact, all of the data of the graph $G$ is captured in the two tables above—together they tell us the sets $A$ and $V$ and the functions $src$ and $tgt.$
\end{exampleENG}

\begin{exampleRUS}[Граф]\label{ex:graph}
Вот изображение графа $G=(V,A,src,tgt)$:
\begin{align}\label{dia:graph}
G:=\parbox{2in}{\fbox{\xymatrix{
    \bullet^v\ar[r]^f&\bullet^w\ar@/_1pc/[r]_h\ar@/^1pc/[r]^g&\bullet^x\\
    \bullet^y\ar@(l,u)[]^i\ar@/^1pc/[r]^j&\bullet^z\ar@/^1pc/[l]^k
}}}
\end{align} 
Здесь у нас $V=\{v,w,x,y,z\}$ и $A=\{f,g,h,i,j,k\}.$ Функции начала и конца $src,tgt\taking A\to V$ задаются таблицей ниже (слева):
\begin{align*}
\begin{array}{| l || l | l |}\bhline
{\bf A}&{\bf src}&{\bf tgt}\\\bbhline
f&v&w\\\hline
g&w&x\\\hline
h&w&x\\\hline
i&y&y\\\hline
j&y&z\\\hline
k&z&y\\\bhline
\end{array}
\hspace{1in}
\begin{array}{| l |}\bhline
{\bf V}\\\bbhline
v\\\hline
w\\\hline
x\\\hline
y\\\hline
z\\\bhline
\end{array}
\end{align*}
В принципе, вся информация о графе $G$ охвачена двумя таблицами выше: вместе они говорят нам все о множествах $A$ и $V$ и функциях $src$ и $tgt.$
\end{exampleRUS}

\begin{exampleENG}
Every olog has an underlying graph. The additional information in an olog has to do with which pairs of paths are declared equivalent, as well as text that has certain English-readability rules.\index{olog!underlying graph}
\end{exampleENG}

\begin{exampleRUS}
Каждый олог имеет подлежащий граф. Дополнительная информация в ологе касается того, какие пары путей объявлены эквивалентными, а также человеко-читаемого текста на естественном языке.\index{олог!подлежащий граф}
\end{exampleRUS}

\begin{exerciseENG}
\sexc Draw the graph corresponding to the following tables:
\begin{align*}
\begin{array}{| l || l | l |}\bhline
{\bf A}&{\bf src}&{\bf tgt}\\\bbhline
f&v&w\\\hline
g&v&w\\\hline
h&v&w\\\hline
i&x&w\\\hline
j&z&w\\\hline
k&z&z\\\bhline
\end{array}
\hspace{1in}
\begin{array}{| l |}\bhline
{\bf V}\\\bbhline
u\\\hline
v\\\hline
w\\\hline
x\\\hline
y\\\hline
z\\\bhline
\end{array}
\end{align*}
\item Write down two tables, as above, corresponding to the following graph:
$$\fbox{\xymatrix{
\LMO{a}\ar[r]^{1}&\LMO{b}\ar[r]^2\ar@/^1pc/[r]^3&\LMO{c}\ar@/^1pc/[l]^4\ar[r]^5&\LMO{d}\\
\LMO{e}&\LMO{f}\ar[l]^6\ar[r]_7&\LMO{g}\ar[ur]_8}}
$$
\endsexc
\end{exerciseENG}

\begin{exerciseRUS}
\sexc Изобразите граф, соответствующий следующим таблицам:
\begin{align*}
\begin{array}{| l || l | l |}\bhline
{\bf A}&{\bf src}&{\bf tgt}\\\bbhline
f&v&w\\\hline
g&v&w\\\hline
h&v&w\\\hline
i&x&w\\\hline
j&z&w\\\hline
k&z&z\\\bhline
\end{array}
\hspace{1in}
\begin{array}{| l |}\bhline
{\bf V}\\\bbhline
u\\\hline
v\\\hline
w\\\hline
x\\\hline
y\\\hline
z\\\bhline
\end{array}
\end{align*}
\item Выпишите две таблицы, подобные упомянутым выше, которые соответствуют следующему графу:
$$
\fbox{\xymatrix{
    \LMO{a}\ar[r]^{1}&\LMO{b}\ar[r]^2\ar@/^1pc/[r]^3&\LMO{c}\ar@/^1pc/[l]^4\ar[r]^5&\LMO{d}\\
    \LMO{e}&\LMO{f}\ar[l]^6\ar[r]_7&\LMO{g}\ar[ur]_8
}}
$$
\endsexc
\end{exerciseRUS}

\begin{exerciseENG}
Let $A=\{1,2,3,4,5\}$ and $B=\{a,b,c\}.$ Draw them and choose an arbitrary function $f\taking A\to B$ and draw it. Let $A\sqcup B$ be the coproduct of $A$ and $B$ (Definition~\ref{def:coproduct}) and let $A\To{i_1}A\sqcup B\From{i_2}B$ be the two inclusions. Consider the two functions $src,tgt\taking A\to A\sqcup B,$ where $src=i_1$ and $tgt$ is the composition $A\To{f}B\To{i_2}A\sqcup B.$ Draw the associated graph $(A\sqcup B,A,src,tgt).$
\end{exerciseENG}

\begin{exerciseRUS}
Пусть $A=\{1,2,3,4,5\}$ и $B=\{a,b,c\}.$ Изобразите их; выберите произвольную функцию $f\taking A\to B$ and и изобразите ее. Пусть $A\sqcup B$ — это копроизведение $A$ и $B$ (Определение~\ref{def:coproduct}), а $A\To{i_1}A\sqcup B\From{i_2}B$ — два вложения. Рассмотрим две функции $src,tgt\taking A\to A\sqcup B,$ где $src=i_1$ и $tgt$ является композицией $A\To{f}B\To{i_2}A\sqcup B.$ Изобразите соответствующий граф $(A\sqcup B,A,src,tgt).$
\end{exerciseRUS}

\begin{exerciseENG}~
\sexc Let $V$ be a set. Suppose we just draw the elements of $V$ as vertices and have no arrows between them. Is this a graph?
\item Given $V,$ is there any other “canonical” or somehow automatic non-random procedure for generating a graph with those vertices? 
\endsexc
\end{exerciseENG}

\begin{exerciseRUS}~
\sexc Пусть $V$ — это множество. Предположим, мы просто нарисовали элементы $V$ в виде вершин безо всяких стрелочек между ними. Получится ли у нас граф?
\item Имеются ли другие «канонические» или в каком-то смысле автоматические, неслучайные процедуры для построения по данному множеству вершин $V$ графа с этими вершинами? 
\endsexc
\end{exerciseRUS}

\begin{exampleENG}
Recall from Construction~\ref{const:bipartite} the notion of bipartite graph, which we defined to be a span (i.e. pair of functions, see Definition~\ref{def:span}) $A\From{f}R\To{g}B.$ Now that we have a formal definition of graph, we might hope that bipartite graphs fit in, and they do. Let $V=A\sqcup B$ and let $i\taking A\to V$ and $j\taking B\to V$ be the inclusions. Let $src=i\circ f\taking R\to V$ and let $tgt=j\circ g\taking R\to V$ be the composites.
$$
\xymatrix{&A\ar[dr]^i\\R\ar@/^1pc/[rr]_{src}\ar@/_1pc/[rr]^{tgt}\ar[ur]^f\ar[dr]_g&&V\\&B\ar[ur]_j}
$$ 
Then $(V,R,src,tgt)$ is a graph that would be drawn exactly as we specified the drawing of spans in Construction~\ref{const:bipartite}.
\end{exampleENG}

\begin{exampleRUS}
Напомним из Конструкции~\ref{const:bipartite} понятие двудольного графа, который определяется как развилка (т.е. пара функций, см. Определение~\ref{def:span}) $A\From{f}R\To{g}B.$ Теперь, когда у нас есть формальное определение графа, мы могли бы надеяться, что двудольный граф будет частным его случаем, и это действительно так. Пусть $V=A\sqcup B,$ а $i\taking A\to V$ и $j\taking B\to V$ — вложения. Положим $src:=i\circ f\taking R\to V$ и $tgt:=j\circ g\taking R\to V$ (заданные как композиции).
$$\xymatrix{
      &  A\ar[dr]^i  \\
    R\ar@/^1pc/[rr]_{src}\ar@/_1pc/[rr]^{tgt}\ar[ur]^f\ar[dr]_g  &  &  V  \\
      &  B\ar[ur]_j
}$$ 
Тогда $(V,R,src,tgt)$ является графом, который изображается точно так, как изображаются развилки согласно Конструкции~\ref{const:bipartite}.
\end{exampleRUS}

\begin{exampleENG}\label{ex:[n] as graph}
Let $n\in\NN$ be a natural number. The {\em chain graph of length $n$},\index{graph!chain} denoted $[n]$ is the graph depicted here:
$$
\xymatrix{
\LMO{0}\ar[r]&\LMO{1}\ar[r]&\cdots\ar[r]&\LMO{n}
}
$$
In general $[n]$ has $n$ arrows and $n+1$ vertices. In particular, when $n=0$ we have that $[0]$ is the graph consisting of a single vertex and no arrows. 
\end{exampleENG}

\begin{exampleRUS}\label{ex:[n] as graph}
Пусть $n\in\NN$ — это натуральное число. {\em Граф-цепь длины $n$},\index{граф!цепь} обозначаемый $[n],$ — это граф, изображенный ниже:
$$\xymatrix{
    \LMO{0}\ar[r]  &  \LMO{1}\ar[r]  &  \cdots\ar[r]  &  \LMO{n}
}$$
В общем случае $[n]$ имеет $n$ стрелок и $n+1$ вершину. В частности, при $n=0$ получается, что граф $[0]$ состоит из единственной вершины и не имеет стрелок. 
\end{exampleRUS}

\begin{exampleENG}\label{ex:ZxG}
Let $G=(V,A,src,tgt)$ be a graph; we want to spread it out over discrete time, so that each arrow does not occur within a given time-slice but instead over a quantum unit of time. 

Let $N=(\NN,\NN,n\mapsto n,n\mapsto n+1)$ be the graph depicted 
$$\xymatrix{\LMO{0}\ar[r]^0&\LMO{1}\ar[r]^1&\LMO{2}\ar[r]^2&\cdots}$$
When we get to limits in a category, we will understand that products can be taken in the category of graphs (see  Example~\ref{ex:product of graphs}), and $N\times G$ will make sense. For now, we construct it by hand.

Let $T(G)=(V\times \NN,A\times\NN,src',tgt')$ be a new graph, where for $a\in A$ and $n\in\NN$ we have $src'(a,n):=(src(a),n)$ and $tgt'(a,n)=(tgt(a),n+1).$ This may be a bit much to swallow, so try to simply understand what is being done in the following example. 

Let $G$ be the graph drawn below 
$$\xymatrix{\LMO{a}\ar@(ul,ur)[]^f\ar[d]_g\\\LMO{b}}$$
Then $T(G)$ will be the graph 
$$\xymatrix@=30pt{
\LMO{a0}\ar[r]^{f0}\ar[rd]_{g0}&\LMO{a1}\ar[r]^{f1}\ar[rd]_{g1}&\LMO{a2}\ar[r]^{f2}\ar[rd]_{g2}&\cdots\\
\LMO{b0}&\LMO{b1}&\LMO{b2}&\cdots
}
$$
As you can see, $f$-arrows still take $a$'s to $a$'s and $g$-arrows still take $a$'s to $b$'s, but they always march forward in time.
\end{exampleENG}

\begin{exampleRUS}\label{ex:ZxG}
Пусть $G=(V,A,src,tgt)$ — это граф; мы хотим разместить его в дискретном времени так, чтобы каждая стрелка возникала не в отдельный момент [нулевой продолжительности], а на протяжении некоторого кванта времени. 

Пусть $N=(\NN,\NN,n\mapsto n,n\mapsto n+1)$ — это граф, изображенный ниже:
$$\xymatrix{
    \LMO{0}\ar[r]^0  &  \LMO{1}\ar[r]^1  &  \LMO{2}\ar[r]^2  &  \cdots
}$$
Когда мы дойдем до понятия пределов в категории, мы осознаем, что в категории графов можно образовывать произведения (см. Пример~\ref{ex:product of graphs}), и что имеет смысл выражение $N\times G.$ Сейчас же мы построим его вручную.

Пусть $T(G)=(V\times \NN,A\times\NN,src',tgt')$ — это новый граф, где для любого $a\in A$ и $n\in\NN$ выполняется $src'(a,n):=(src(a),n)$ и $tgt'(a,n):=(tgt(a),n+1).$ Возможно, это немного трудно переварить сразу, так что попытаемся понять, что происходит, на следующем примере. 

Пусть $G$ — это граф, изображенный ниже:
$$\xymatrix{
    \LMO{a} \ar@(ul,ur)[]^f \ar[d]_g  \\
    \LMO{b}
}$$
Тогда $T(G)$ будет графом
$$\xymatrix@=30pt{
    \LMO{a0} \ar[r]^{f0} \ar[rd]_{g0}  &  \LMO{a1} \ar[r]^{f1} \ar[rd]_{g1}  &  \LMO{a2} \ar[r]^{f2} \ar[rd]_{g2}  &  \cdots  \\
    \LMO{b0}  &  \LMO{b1}  &  \LMO{b2}  &  \cdots
}$$
Как вы видите, $f$-стрелки по прежнему идут из $a$-вершин в $a$-вершины, а $g$-стрелки — из $a$-вершин в $b$-вершины, при этом они всегда двигаются по времени вперед.
\end{exampleRUS}

\begin{exerciseENG}\label{exc:secret turing}
Let $G$ be the graph depicted below:
$$
\xymatrix{\LMO{a}\ar@/^1pc/[rr]^w\ar@(lu,ld)[]_x&&\LMO{b}\ar@/^1pc/[ll]^y\ar@(ur,dr)[]^z}
$$
Draw (using ellipses “$\cdots$” if necessary) the graph $T(G)$ defined in Example~\ref{ex:ZxG}.
\end{exerciseENG}

\begin{exerciseRUS}\label{exc:secret turing}
Пусть $G$ — это граф, изображенный ниже:
$$\xymatrix{
    \LMO{a} \ar@/^1pc/[rr]^w \ar@(lu,ld)[]_x  &  &  \LMO{b} \ar@/^1pc/[ll]^y \ar@(ur,dr)[]^z
}$$
Нарисуйте (используя по необходимости «$\cdots$») граф $T(G),$ определенный в Примере~\ref{ex:ZxG}.
\end{exerciseRUS}

\begin{exerciseENG}\label{exc:lattice}
Consider the infinite graph $G=(V,A,src,tgt)$ depicted below,
$$
\xymatrix{
\vdots&\vdots&\vdots\\
(0,2)\ar[r]\ar[u]&(1,2)\ar[r]\ar[u]&(2,2)\ar[r]\ar[u]&\cdots\\
(0,1)\ar[r]\ar[u]&(1,1)\ar[r]\ar[u]&(2,1)\ar[r]\ar[u]&\cdots\\
(0,0)\ar[r]\ar[u]&(1,0)\ar[r]\ar[u]&(2,0)\ar[r]\ar[u]&\cdots}
$$
\sexc Write down the sets $A$ and $V.$ 
\item What are the source and target function $A\to V?$  
\endsexc
\end{exerciseENG}

\begin{exerciseRUS}\label{exc:lattice}
\end{exerciseRUS}

\begin{exerciseENG}\label{exc:(co)equalizer of graph}
A graph is a pair of functions $A\tto V.$ This sets up the notion of equalizer and coequalizer (see Definitions~\ref{def:equalizer} and~\ref{def:coequalizer}). 
\sexc What feature of a graph is captured by the equalizer of its source and target functions? 
\item What feature of a graph is captured by the coequalizer of its source and target functions?
\endsexc
\end{exerciseENG}

\begin{exerciseRUS}\label{exc:(co)equalizer of graph}
\end{exerciseRUS}

%%%% Subsection %%%%

\subsection{\caseENGRUS{Paths in a graph}{ / }{Пути в графе}}\label{sec:paths in graph}

\begin{blockENG}
We all know what a path in a graph is, especially if we understand that a path must always follow the direction of arrows. The following definition makes this idea precise. In particular, one can have paths of any finite length $n\in\NN,$ even length $0$ or $1.$ Also, we want to be able to talk about the source vertex and target vertex of a path, as well as concatenation of paths.
\end{blockENG}

\begin{blockRUS}
\end{blockRUS}

\begin{definitionENG}\label{def:paths in graph}\index{graph!paths}
Let $G=(V,A,src,tgt)$ be a graph. A {\em path of length $n$}\index{path} in $G,$ denoted $p\in\Path_G^{(n)}$\index{a symbol!$\Path$} is a head-to-tail sequence \begin{align}\label{dia:path}p=(v_0\To{a_1}v_1\To{a_2}v_2\To{a_3}\ldots\To{a_n}v_n)\end{align} of arrows in $G,$ which we denote by $v_0 a_1 a_2 \ldots a_n.$ In particular we have canonical isomorphisms $\Path_G^{(1)}\iso A$ and $\Path_G^{(0)}\iso V$; we refer to the path of length 0 on vertex $v$ as the {\em trivial path on $v$} and denote it simply by $v.$ We denote by $\Path_G$ the set of paths in $G,$ $$\Path_G:=\bigcup_{n\in\NN}\Path_G^{(n)}.$$ Every path $p\in\Path_G$ has a source vertex and a target vertex, and we may denote these by $\ol{src},\ol{tgt}\taking\Path_G\to V.$ If $p$ is a path with $\ol{src}(p)=v$ and $\ol{tgt}(p)=w,$ we may denote it by $p\taking v\to w.$ Given two vertices $v,w\in V,$ we write $\Path_G(v,w)$ to denote the set of all paths $p\taking v\to w.$

There is a concatenation operation on paths.\index{concatenation!of paths} Given a path $p\taking v\to w$ and $q\taking w\to x,$ we define the concatenation, denoted $p q\taking v\to x$ in the obvious way. If $p=va_1,a_2\ldots a_m$ and $q= wb_1b_2\ldots b_n$ then $pq=va_1\ldots a_mb_1\ldots b_n.$ In particular, if $p$ (resp. $r$) is the trivial path on vertex $v$ (resp. vertex $w$) then for any path $q\taking v\to w,$ we have $pq=q$ (resp. $qr=q$). 
\end{definitionENG}

\begin{definitionRUS}\label{def:paths in graph}\index{граф!пути}
\end{definitionRUS}

\begin{exampleENG}
In Diagram (\ref{dia:graph}), page \pageref{dia:graph}, there are no paths from $v$ to $y,$ one path ($f$) from $v$ to $w,$ two paths ($fg$ and $fh$) from $v$ to $x,$ and infinitely many paths $$\{y i^{p_1}(jk)^{q_1}\cdots i^{p_n}(jk)^{q_n}\;|\;n,p_1,q_1,\ldots,p_n,q_n\in\NN\}$$ from $y$ to $y.$ There are other paths as well, including the five trivial paths.
\end{exampleENG}

\begin{exampleRUS}
\end{exampleRUS}

\begin{exerciseENG}
How many paths are there in the following graph? 
$$\xymatrix{\LMO{1}\ar[r]^{f}&\LMO{2}\ar[r]^{g}&\LMO{3}}$$
\end{exerciseENG}

\begin{exerciseRUS}
\end{exerciseRUS}

\begin{exerciseENG}
Let $G$ be a graph and consider the set $\Path_G$ of paths in $G.$ Suppose someone claimed that there is a monoid structure on the set $\Path_G,$ where the multiplication formula is given by concatenation of paths. Are they correct? Why or why not? Hint: what should be the identity element?
\end{exerciseENG}

\begin{exerciseRUS}
\end{exerciseRUS}

%%%% Subsection %%%%

\subsection{\caseENGRUS{Graph homomorphisms}{ / }{Гомоморфизмы графов}}

\begin{blockENG}
A graph $(V,A,src,tgt)$ involves two sets and two functions. For two graphs to be comparable, their two sets and their two functions should be appropriately comparable.\index{appropriate comparison}
\end{blockENG}

\begin{blockRUS}
\end{blockRUS}

\begin{definitionENG}\label{def:graph homomorphism}\index{graph!homomorphism}
Let $G=(V,A,src,tgt)$ and $G'=(V',A',src',tgt')$ be graphs. A {\em graph homomorphism $f$ from $G$ to $G'$}, denoted $f\taking G\to G',$ consists of two functions $f_0\taking V\to V'$ and $f_1\taking A\to A'$ such that the two diagrams below commute:
\begin{align}\label{dia:graph hom}
\xymatrix{A\ar[r]^{f_1}\ar[d]_{src}&A'\ar[d]^{src'}\\V\ar[r]_{f_0}&V'
}\hspace{1in}
\xymatrix{A\ar[r]^{f_1}\ar[d]_{tgt}&A'\ar[d]^{tgt'}\\V\ar[r]_{f_0}&V'
}
\end{align}
\end{definitionENG}

\begin{definitionRUS}\label{def:graph homomorphism}\index{graph!homomorphism}
\end{definitionRUS}

\begin{remarkENG}
The above conditions (\ref{dia:graph hom}) may look abstruse at first, but they encode a very important idea, roughly stated “arrows are bound to their vertices”. Under a map of graphs $G\to G'$ , one cannot flippantly send an arrow of $G$ any old arrow of $G'$: it must still connect the vertices it connected before. Below is an example of a mapping that does not respect this condition: $a$ connects $1$ and $2$ before, but not after:
$$
\fbox{\xymatrix{\LMO{\color{red}{1}}\ar[r]^{a}&\LMO{\color{blue}{2}}}}
\xymatrix{~\ar[rr]^{1\mapsto 1',2\mapsto 2', a\mapsto a'}&\hsp&~}
\fbox{\xymatrix{\LMO{\color{red}{1'}}&\LMO{\color{blue}{2'}}\ar[r]^{a'}&\LMO{\color{ForestGreen}{3'}}}}
$$
The commutativity of the diagrams in (\ref{dia:graph hom}) is exactly what is needed to ensure that arrows are handled in the expected way by a proposed graph homomorphism.
\end{remarkENG}

\begin{remarkRUS}
\end{remarkRUS}

\begin{exampleENG}[Graph homomorphism]\label{ex:graph hom}
Let $G=(V,A,src,tgt)$ and $G'=(V',A',src',tgt')$ be the graphs drawn to the left and right (respectively) below:
\begin{align}\label{dia:graph hom example}
\parbox{1.5in}{\fbox{\xymatrix{\LMO{\color{red}{1}}\ar[r]^a\ar@/^1pc/[d]^d\ar@/_1pc/[d]_c&\LMO{\color{ForestGreen}{2}}\ar[r]^b&\LMO{\color{red}{3}}\\\LMO{4}&\LMO{\color{blue}{5}}\ar[r]^e&\LMO{\color{blue}{6}}}}}
\parbox{1in}{\xymatrix{~\ar[rr]^{\parbox{.8in}{\vspace{-.2in}\footnotesize$1\mapsto 1', 2\mapsto 2',\\ 3\mapsto 1',4\mapsto 4',\\ 5\mapsto 5',6\mapsto5'$}}&\hsp&~}}
\parbox{.8in}{\fbox{\xymatrix{\LMO{\color{red}{1'}}\ar@<.5ex>[r]^w\ar[d]_y&\LMO{\color{ForestGreen}{2'}}\ar@<.5ex>[l]^x\\\LMO{4'}&\LMO{\color{blue}{5'}}\ar@(r,u)[]_z}}}
\end{align}
The colors indicate our choice of function $f_0\taking V\to V'.$ Given that choice, condition (\ref{dia:graph hom}) imposes in this case that there is a unique choice of graph homomorphism $f\taking G\to G'.$ 
\end{exampleENG}

\begin{exampleRUS}[Graph homomorphism]\label{ex:graph hom}
\end{exampleRUS}

\begin{exerciseENG}~
\sexc Where are $a,b,c,d,e$ sent under $f_1\taking A\to A'$ in Diagram (\ref{dia:graph hom example})? 
\item Choose a couple elements of $A$ and check that they behave as specified by Diagram (\ref{dia:graph hom}).
\endsexc
\end{exerciseENG}

\begin{exerciseRUS}~
\end{exerciseRUS}

\begin{exerciseENG}
Let $G$ be a graph, let $n\in\NN$ be a natural number, and let $[n]$ be the chain graph of length $n,$ as in Example~\ref{ex:[n] as graph}. Is a path of length $n$ in $G$ the same thing as a graph homomorphism $[n]\to G,$ or are there subtle differences? More precisely, is there always an isomorphism between the set of graph homomorphisms $[n]\to G$ and the set $\Path_G^{(n)}$ of length-$n$ paths in $G?$
\end{exerciseENG}

\begin{exerciseRUS}
\end{exerciseRUS}

\begin{exerciseENG}
Given a morphism of graphs $f\taking G\to G',$ there an induced function $\Path(f)\taking\Path(G)\to\Path(G').$ 
\sexc Is it the case that for every $n\in\NN,$ the function $\Path(f)$ carries $\Path^{(n)}(G)$ to $\Path^{(n)}(G'),$ or can path lengths change in this process?
\item Suppose that $f_0$ and $f_1$ are injective (meaning no two distinct vertices in $G$ are sent to the same vertex (respectively for arrows) under $f$). Does this imply that $\Path(f)$ is also injective (meaning no two distinct paths are sent to the same path under $f$)?
\item Suppose that $f_0$ and $f_1$ are surjective (meaning every vertex in $G'$ and every arrow in $G'$ is in the image of $f$). Does this imply that $\Path(f)$ is also surjective? Hint: at least one of the answers to these three questions is “no”.
\endsexc
\end{exerciseENG}

\begin{exerciseRUS}
\end{exerciseRUS}

\begin{exerciseENG}\label{exc:single condition for graph hom}
Given a graph $(V,A,src,tgt),$ let $i\taking A\to V\times V$ be function guaranteed by the universal property for products, as applied to $src,tgt\taking A\to V.$ One might hope to summarize Condition (\ref{dia:graph hom}) for graph homomorphisms by the commutativity of the single square 
\begin{align}\label{dia:equiv graph hom}
\xymatrix{A\ar[r]^{f_1}\ar[d]_{i}&A'\ar[d]^{i'}\\V\times V\ar[r]_{f_0\times f_0}&V'\times V'.}
\end{align}
Is the commutativity of the diagram in (\ref{dia:equiv graph hom}) indeed equivalent to the commutativity of the diagrams in (\ref{dia:graph hom})?
\end{exerciseENG}

\begin{exerciseRUS}\label{exc:single condition for graph hom}
\end{exerciseRUS}

%% Subsubsection %%

\subsubsection{\caseENGRUS{Binary relations and graphs}{ / }{Бинарные отношения и графы}}

\begin{definitionENG}\label{def:binary relation}\index{relation!binary}
Let $X$ be a set. A {\em binary relation on $X$} is a subset $R\ss X\times X.$ 
\end{definitionENG}

\begin{definitionRUS}\label{def:binary relation}\index{relation!binary}
\end{definitionRUS}

\begin{blockENG}
If $X=\NN$ is the set of integers, then the usual $\leq$ defines a relation on $X$: given $(m,n)\in\NN\times\NN,$ we put $(m,n)\in R$ iff $m\leq n.$ As a table it might be written as to the left
\begin{align}\label{dia:3 relations}
\begin{tabular}{|p{.7cm}|p{.7cm}|}
\bhline
\multicolumn{2}{|c|}{$m\leq n$}\\\bhline
m&n\\\bbhline
0&0\\\hline
0&1\\\hline
1&1\\\hline
0&2\\\hline
1&2\\\hline
2&2\\\hline
0&3\\\bhline
$\vdots$&$\vdots$\\\hline
\end{tabular}
\hspace{1in}
\begin{tabular}{|p{.7cm}|p{.7cm}|}
\bhline
\multicolumn{2}{|c|}{$n=5m$}\\\bhline
m&n\\\bbhline
0&0\\\hline
1&5\\\hline
2&10\\\hline
3&15\\\hline
4&20\\\hline
5&25\\\hline
6&30\\\bhline
$\vdots$&$\vdots$\\\hline
\end{tabular}
\hspace{1in}
\begin{tabular}{|p{.7cm}|p{.7cm}|}
\bhline
\multicolumn{2}{|c|}{$|n-m|\leq 1$}\\\bhline
m&n\\\bbhline
0&0\\\hline
0&1\\\hline
1&0\\\hline
1&1\\\hline
1&2\\\hline
2&1\\\hline
2&2\\\hline
$\vdots$&$\vdots$\\\hline
\end{tabular}
\end{align}
The middle table is the relation $\{(m,n)\in\NN\times\NN\|n=5m\}\ss\NN\times\NN$ and the right-hand table is the relation $\{(m,n)\in\NN\times\NN\||n-m|\leq 1\}\ss\NN\times\NN.$ 
\end{blockENG}

\begin{blockRUS}
\end{blockRUS}

\begin{exerciseENG}
A relation on $\RR$ is a subset of $\RR\times\RR,$ and one can indicate such a subset of the plane by shading. Choose an error bound $\epsilon>0$ and draw the relation one might refer to as “$\epsilon$-approximation”. To say it another way, draw the relation “$x$ is within $\epsilon$ of $y$”.
\end{exerciseENG}

\begin{exerciseRUS}
\end{exerciseRUS}

\begin{exerciseENG}[Binary relations to graphs]\label{exc:rel to graph}\index{relation!graph of}
\sexc If $R\ss S\times S$ is a binary relation, find a natural way to make a graph out of it, having vertices $S.$ 
\item What is the set $A$ of arrows? 
\item What are the source and target functions $src,tgt\taking A\to S?$
\item Take the left-hand table in (\ref{dia:3 relations}) and consider its first $7$ rows (i.e. forget the $\vdots$). Draw the corresponding graph (do you see a tetrahedron?). 
\item Do the same for the right-hand table.
\endsexc
\end{exerciseENG}

\begin{exerciseRUS}[Binary relations to graphs]\label{exc:rel to graph}\index{relation!graph of}
\end{exerciseRUS}

\begin{exerciseENG}[Graphs to binary relations]\label{ex:graph to rel}~
\sexc If $(V,A,src,tgt)$ is a graph, find a natural way to make a binary relation $R\ss V\times V$ out of it. 
\item Take the left-hand graph $G$ from (\ref{dia:graph hom example}) and write out the corresponding binary relation in table form.
\endsexc
\end{exerciseENG}

\begin{exerciseRUS}[Graphs to binary relations]\label{ex:graph to rel}~
\end{exerciseRUS}

\begin{exerciseENG}[Going around the loops]
\sexc Given a binary relation $R\ss S\times S,$ you know from Exercise~\ref{exc:rel to graph} how to construct a graph out of it, and from Exercise~\ref{ex:graph to rel} how to make a new binary relation out of that. How does the resulting relation compare with the original?
\item Given a graph $(V,A,src,tgt),$ you know from Exercise~\ref{ex:graph to rel} how to make a new binary relation out of it, and from Exercise~\ref{exc:rel to graph} how to construct a new graph out of that. How does the resulting graph compare with the original? 
\endsexc
\end{exerciseENG}

\begin{exerciseRUS}[Going around the loops]
\end{exerciseRUS}

\end{document}
