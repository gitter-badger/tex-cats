\documentclass[../main/CT4S-EN-RU]{subfiles}

\begin{document}

\section*{\caseENGRUS{Preamble}{ / }{Преамбула}}

\begin{blockENG}
In this chapter we begin to use our understanding of sets to build more interesting mathematical devices, each of which organizes our understanding of a certain kind of domain. For example, monoids organize our thoughts about agents acting on objects; groups are monoids except restricted to only allow agents to act reversibly. We will then study graphs, which are systems of nodes and arrows that can capture ideas like information flow through a network or model connections between building blocks in a material. We will discuss orders, which can be used to study taxonomies or hierarchies. Finally we take a mathematical look at databases, which actually subsume everything else in the chapter. Databases are connection patterns for structuring information.
\end{blockENG}

\begin{blockRUS}
В этой главе мы используем наше понимание множеств для построения более интересных математических механизмов, каждый из которых вносит порядок в наше понимание определенной области. Например, моноиды организуют наши мысли о действии на объекты влияющих факторов; группы это моноиды, только факторы должны действовать обратимо. Затем мы изучим графы, которые являются системами вершин и стрелок, которые ухватывают идеи переноса информации через сеть или моделирование связей между строительными блоками в материале. Мы обсудим порядки, которые используют для изучения таксономий или иерархий. Наконец, мы математическим способом взглянем на базы данных, которые на самом деле подводят итог всему остальному в этой главе. Базы данных являются связующим материалом в структурировании информации.
\end{blockRUS}

\begin{blockENG}
We will see in Chapter~\ref{chap:categories} that everything we study in the present chapter is an example of a category. So is $\Set,$ the category of sets studied in Chapter~\ref{chap:sets}. One way to think of a category is as a set of objects and a connection pattern between them; sets are objects (ovals full of dots if you wish) connected by functions. But each set is itself a category: the objects inside it are just disconnected! Just like a set has an interior view and an exterior view, so will all the categories in this chapter. Each monoid {\em is} a category, but there is also a category {\em of} monoids. 
\end{blockENG}

\begin{blockRUS}
В Главе~\ref{chap:categories} мы увидим, что все понятия, изученные в данной главе, являются примерами категорий. Ею же является и $\Set,$ категория множеств, изученная в Главе~\ref{chap:sets}. О категории можно думать как о множестве объектов и о связующем материале между ними; множества являются объектами (овалы, заполненные точками, если мы пожелаем [их так вообразить]), и они связаны между собой функциями. Но каждое множество само является категорией: объекты внутри просто отсоединены друг от друга! И подобно тому, как множества имеют вид изнутри и снаружи, так же и остальные категории в этой главе [образуют категории в совокупности и сами по себе являются категориями]. Например, каждый моноид это категория, но имеется также и категория моноидов. 
\end{blockRUS}

\begin{blockENG}
However, we will not really say the word “category” much if at all in this chapter. It seems preferable to let the ideas rise on their own accord as interesting structures in their own right before explaining that everything in site fits into a single framework. That will be the pleasant reward to come in Chapter~\ref{chap:categories}.
\end{blockENG}

\begin{blockRUS}
Тем не менее, мы не будем часто произносить в этой главе слово «категория», если вообще будем. Лучше дать идеям прозвучать самостоятельно как самостоятельным интересным структурам перед объяснением того, что все рассмотренное входит в единый математический аппарат. Тем приятнее будет сделать последнее в Главе~\ref{chap:categories}.
\end{blockRUS}

\end{document}
