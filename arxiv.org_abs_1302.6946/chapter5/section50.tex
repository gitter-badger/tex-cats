\documentclass[../main/CT4S-EN-RU]{subfiles}

\begin{document}

\section*{\caseENGRUS{Preamble}{ / }{Преамбула}}

\begin{blockENG}
We have now set up an understanding of the basic notions of category theory: categories, functors, natural transformations, and universal properties. We have discussed many sources of examples: orders, graphs, monoids, and databases. We begin this chapter with the notion of {\em adjoint functors} (also known as {\em adjunctions}), which are like dictionaries that translate back and forth between different categories. 
\end{blockENG}

\begin{blockRUS}
Итак, мы добились понимания основных понятий теории категорий: категорий, функторов, естественных преобразований и универсальных свойств. Мы обсудили многочисленные источники примеров: порядки, графы, моноиды и базы данных. Эту главу мы начнем с понятия {\em сопряженных функторов} (также известных как {\em сопряжения}), которые подобны словарям, позволяющим осуществлять перевод между различными категориями в обоих направлениях. 
\end{blockRUS}

\end{document}
