\documentclass[../main/CT4S-EN-RU]{subfiles}

\begin{document}

\section{\caseENGRUS{Category theory references}{ / }{Литература по теории категорий}}

\begin{blockENG}
I wrote this book because the available books on category theory are almost all written for mathematicians (the rest are written for computer scientists). There is one book by Lawvere and Schanuel, called {\em Conceptual Mathematics} \cite{LS}, that offers category theory to a wider audience, but its style is not appropriate for this course. Still, it is very well written and clear.
\end{blockENG}

\begin{blockRUS}
Я написал эту книгу, потому что имеющиеся книги по теории категорий практически все написаны для математиков (остальные написаны для теоретиков информатики). Есть одна книга Ловера и Шануеля {\em Концептуальная математика} \cite{LS}, которая предлагает теорию категорий более широкой аудитории, но ее стиль не подходит для этого курса. Тем не менее, написана она очень хорошо и ясно. 
\end{blockRUS}

\begin{blockENG}
The “bible” of category theory is {\em Categories for the working mathematician} by Mac Lane \cite{Mac}. But as the title suggests, it was written for working mathematicians and will be quite opaque to my target audience. However, once a person has read my book, Mac Lane's book may become a valuable reference.
\end{blockENG}

\begin{blockRUS}
«Библия» теории категорий это {\em Категории для работающего математика} Мак Лейна \cite{Mac}. Но, как и предполагает ее заглавие, написана она для профессиональных математиков и будет совершенно непонятна моей целевой аудитории. Однако, после прочтения моей книги, книга Мак Лейна может стать ценным справочником.  
\end{blockRUS}

\begin{blockENG}
Other good books include Steve Awodey's book {\em Category theory} \cite{Awo} and Barr and Wells book {\em Category theory for computing science}, \cite{BW}.  A paper by Brown and Porter called  \href{http://pages.bangor.ac.uk/\%7Emas010/pdffiles/Analogy-and-Comparison.pdf}{\text Category Theory: an abstract setting for analogy and comparison} \cite{BP1} is more in line with the style of this book, only much shorter. Online, I find \href{http://www.wikipedia.org}{\text wikipedia} and a site called \href{http://ncatlab.org/nlab/show/HomePage}{\em the $n$lab} to be quite useful.
\end{blockENG}

\begin{blockRUS}
Другие хорошие книги включают книгу Стива Эводи {\em Теория категорий} \cite{Awo}, а также книгу Барра и Уэллса {\em Теория категорий для компьютерных наук}, \cite{BW}. Статья Брауна и Портера под названием  \href{http://pages.bangor.ac.uk/\%7Emas010/pdffiles/Analogy-and-Comparison.pdf}{\text Теория категорий: абстрактный инструмент для аналогий и сравнений} \cite{BP1} более соответствует стилю данной книги, но она значительно короче. Из онлайн-ресурсов, я считаю \href{http://www.wikipedia.org}{\text Википедию} и сайт под названием \href{http://ncatlab.org/nlab/show/HomePage}{$n$Lab} достаточно полезными. 
\end{blockRUS}

\begin{blockENG}
This book attempts to explain category theory by examples and exercises rather than by theorems and proofs. I hope this approach will be valuable to the working scientist.
\end{blockENG}

\begin{blockRUS}
Данная книга пытается изложить теорию категорий при помощи примеров и упражнений, а не теорем и доказательств. Я надеюсь, подобный подход окажется ценным для работающего ученого. 
\end{blockRUS}

\end{document}
