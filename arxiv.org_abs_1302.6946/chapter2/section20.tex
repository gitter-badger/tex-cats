\documentclass[../main/CT4S-EN-RU]{subfiles}

\begin{document}

\section*{\caseENGRUS{Preamble}{ / }{Преамбула}}

\begin{blockENG}
The theory of sets was invented as a foundation for all of mathematics. The notion of sets and functions serves as a basis on which to build our intuition about categories in general. In this chapter we will give examples of sets and functions and then move on to discuss commutative diagrams. At this point we can introduce ologs which will allow us to use the language of category theory to speak about real world concepts. Then we will introduce limits and colimits, and their universal properties. All of this material is basic set theory, but it can also be taken as an investigation of our first category, the {\em category of sets}, which we call $\Set.$ We will end this chapter with some other interesting constructions in $\Set$ that do not fit into the previous sections.
\end{blockENG}

\begin{blockRUS}
Теория множеств разрабатывалась в качестве оснований всей математики. Понятие множества и функции служат основой, на которой строится наша интуиция о категориях вообще. В этой главе мы дадим примеры множеств и функций и затем перейдем к обсуждению коммутативных диаграмм. К тому моменту мы сможем ввести ологи, которые позволят нам использовать язык теории категорий, чтобы говорить о понятиях реального мира. Затем мы введем пределы и копределы и их универсальные свойства. Весь этот материал входит в область элементарной теории множеств, но он может также рассматриваться и как исследование нашей первой категории, {\em категории множеств}, которую мы назовем $\Set.$ Мы закончим эту главу некоторыми другими интересными конструкциями в $\Set,$ которые не подходят к предыдущим разделам.%
\endnote{
TODO ...(возможно, пара слов о логической парадигме более конкретно: нужны ли нам понятия о предикатах и кванторах?)...
}
\end{blockRUS}

\end{document}
