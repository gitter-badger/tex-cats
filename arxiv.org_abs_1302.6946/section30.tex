% section30.tex

\chapter{\caseENGRUS{Categories and functors, without admitting it}{ / }{Категории и функторы: примеры раньше определений}}\label{chap:categories and functors without admitting it}

In this chapter we begin to use our understanding of sets to build more interesting mathematical devices, each of which organizes our understanding of a certain kind of domain. For example, monoids organize our thoughts about agents acting on objects; groups are monoids except restricted to only allow agents to act reversibly. We will then study graphs, which are systems of nodes and arrows that can capture ideas like information flow through a network or model connections between building blocks in a material. We will discuss orders, which can be used to study taxonomies or hierarchies. Finally we take a mathematical look at databases, which actually subsume everything else in the chapter. Databases are connection patterns for structuring information.

We will see in Chapter \ref{chap:categories} that everything we study in the present chapter is an example of a category. So is $\Set$, the category of sets studied in Chapter \ref{chap:sets}. One way to think of a category is as a set of objects and a connection pattern between them; sets are objects (ovals full of dots if you wish) connected by functions. But each set is itself a category: the objects inside it are just disconnected! Just like a set has an interior view and an exterior view, so will all the categories in this chapter. Each monoid {\em is} a category, but there is also a category {\em of} monoids. 

However, we will not really say the word “category” much if at all in this chapter. It seems preferable to let the ideas rise on their own accord as interesting structures in their own right before explaining that everything in site fits into a single framework. That will be the pleasant reward to come in Chapter \ref{chap:categories}.

