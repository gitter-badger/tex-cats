\documentclass[CT4S-EN-RU]{subfiles}

\begin{document}

%\bibliographystyle{amsalpha}
\begin{thebibliography}{SGWB}\rr

\bibitem [Ati]{Ati} Atiyah, M. (1989) “Topological quantum field theories”. {\em Publications Math\'{e}matiques de l'IH\'{E}S} 68 (68), pp. 175--186.

\bibitem [Axl]{Axl} Axler, S. (1997) {\em Linear algebra done right}. Springer. 

\bibitem [Awo]{Awo} S. Awodey. (2010) {\em Category theory.} Second edition. Oxford Logic Guides, 52. Oxford University Press, Oxford.

\bibitem [Bar]{Bar} Bralow, H. (1961) “Possible principles underlying the transformation of sensory messages”. {\em Sensory communication}, pp. 217 -- 234.

\bibitem [BD]{BD} Baez, J.C.; Dolan, J. (1995) “Higher-dimensional algebra and topological quantum field theory”. {\em Journal of mathematical physics} vol 36, 6073.

\bibitem [BFL]{BFL} Baez, J.C.; Fritz, T.; Leinster, T. (2011) “A characterization of entropy in terms of information loss.” {\em Entropy} 13, no. 11.

\bibitem[BS]{BS} Baez, J.C.; Stay, M. (2011) “Physics, topology, logic and computation: a Rosetta Stone.” {\em New structures for physics}, 95–172. Lecture Notes in Phys., 813, Springer, Heidelberg.

\bibitem [BP1]{BP1} Brown, R.; Porter, T. (2006) “Category Theory: an abstract setting for
analogy and comparison, In: {\em What is Category Theory?} Advanced
Studies in Mathematics and Logic, Polimetrica Publisher, Italy, pp. 257-274.

\bibitem [BP2]{BP2} Brown, R.; Porter, T. (2003) “Category theory and higher dimensional
algebra: potential descriptive tools in neuroscience”, {\em Proceedings
of the International Conference on Theoretical Neurobiology, Delhi}, edited by Nandini Singh, National Brain Research
Centre, Conference Proceedings 1 80-92. 

\bibitem [BW]{BW} M. Barr, C. Wells. (1990) {\em Category theory for computing science.} Prentice Hall International Series in Computer Science. Prentice Hall International, New York.

\bibitem [Big]{Big} Biggs, N.M. (2004) {\em Discrete mathematics}. Oxford University Press, NY. 

\bibitem [Dia]{Dia} Diaconescu, R. (2008) {\em Institution-independent model theory} Springer.

\bibitem[DI]{DI} D\"{o}ring, A.; Isham, C. J. “A topos foundation for theories of physics. I. Formal languages for physics.” 
{\em J. Math. Phys.} 49 (2008), no. 5, 053515.

\bibitem[EV]{EV} Ehresmann, A.C.; Vanbremeersch, J.P. (2007) {\em Memory evolutive systems; hierarchy, emergence, cognition}. Elsevier.

\bibitem[Eve]{Eve} Everett III, H. (1973). “The theory of the universal wave function.” In {\em The many-worlds interpretation of quantum mechanics} (Vol. 1, p. 3).

\bibitem [Gog]{Gog} Goguen, J. (1992) “Sheaf semantics for concurrent interacting objects” {\em Mathematical structures in Computer Science} Vol 2, pp. 159 -- 191.

\bibitem [Gro]{Gro} Grothendieck, A. (1971). {\em S\'eminaire de G\'eom\'etrie Alg\'ebrique du Bois Marie - 1960-61 - Rev\^etements \'etales et groupe fondamental - (SGA 1)} (Lecture notes in mathematics 224) (in French). Berlin; New York: Springer-Verlag.

\bibitem [Kro]{Kro} Kr\"{o}mer, R. (2007). {\em Tool and Object: A History and Philosophy of Category Theory}, Birkhauser.

\bibitem [Lam]{Lam} Lambek, J. (1980) “From $\lambda$-calculus to Cartesian closed categories”. In {\em Formalism}, Academic Press, London, pp. 375 -- 402.

\bibitem [Law]{Law} Lawvere, F.W. (2005) “An elementary theory of the category of sets (long version) with
   commentary.” (Reprinted and expanded from Proc. Nat. Acad. Sci. U.S.A. {\bf 52}
   (1964)) {\em Repr. Theory Appl. Categ.} 11, pp. 1 -- 35.
   
\bibitem [Kho]{Kho} Khovanov, M. (2000) “A categorificiation of the Jones polynomial” {\em Duke Math J.}.

\bibitem [Le1]{Le1} Leinster, T. (2004) {\em Higher Operads, Higher Categories}. London Mathematical Society Lecture Note Series 298, Cambridge University Press.

\bibitem [Le2]{Le2} Leinster, T. (2012) “Rethinking set theory”. ePrint available \url{http://arxiv.org/abs/1212.6543}.

\bibitem [Lin]{Lin} Linsker, R. (1988) “Self-organization in a perceptual network”. {\em Computer} 21, no. 3, pp. 105 -- 117.

\bibitem [LM]{LM} Landry, E.; Marquis, J-P., 2005, “Categories in Contexts: historical, foundational, and philosophical.” {\em Philosophia Mathematica}, (3), vol. 13, no. 1, 1 -- 43.

\bibitem [LS]{LS} F.W. Lawvere, S.H. Schanuel. (2009) {\em Conceptual mathematics. 
A first introduction to categories.} Second edition. Cambridge University Press, Cambridge.

\bibitem [MacK]{MacK} MacKay, D.J. (2003). {\em Information theory, inference and learning algorithms.} Cambridge university press.

\bibitem [Mac]{Mac} Mac Lane, S. (1998) {\em Categories for the working mathematician.} Second edition. Graduate Texts in Mathematics, 5. Springer-Verlag, New York.

\bibitem[Mar1]{Mar1} Marquis, J-P. (2009) {\em From a Geometrical Point of View: a study in the history and philosophy of category theory}, Springer.

\bibitem [Mar2]{Mar2} Marquis, J-P, “Category Theory”, {\em The Stanford Encyclopedia of Philosophy} (Spring 2011 Edition), Edward N. Zalta (ed.), \url{http://plato.stanford.edu/archives/spr2011/entries/category-theory}

\bibitem[Min]{Min} Minsky, M. {\em The Society of Mind.}  Simon and Schuster, NY 1985.

\bibitem[Mog]{Mog} Moggi, E. (1989) “A category-theoretic account of program modules.” {\em Category theory and computer science (Manchester, 1989),} 101–117, Lecture Notes in Comput. Sci., 389, Springer, Berlin. 

\bibitem [nLa]{nLa} nLab authors.  \url{http://ncatlab.org/nlab/show/HomePage}

\bibitem [Pen]{Pen} Penrose, R. (2006) {\em The road to reality}. Random house.

\bibitem [RS]{RS} Radul, A.; Sussman, G.J. (2009). “The art of the propagator”. {\em MIT Computer science and artificial intelligence laboratory technical report.}

\bibitem [Sp1]{Sp1} Spivak, D.I. (2012) “Functorial data migration”. {\em Information and communication} 

\bibitem [Sp2]{Sp2} Spivak, D.I. (2012) “Queries and constraints via lifting problems”. Submitted to {\em Mathematical structures in computer science}. ePrint available: \url{http://arxiv.org/abs/1202.2591}

\bibitem [Sp3]{Sp3} Spivak, D.I. (2012) “Kleisli database instances”. ePrint available: \url{http://arxiv.org/abs/1209.1011}

\bibitem [Sp4]{Sp4} Spivak, D.I. (2013) “The operad of wiring diagrams: Formalizing a graphical language for databases, recursion, and plug-and-play circuits”. Available online: \url{http://arxiv.org/abs/1305.0297}

\bibitem[SGWB]{SGWB} Spivak D.I., Giesa T., Wood E., Buehler M.J. (2011) “Category Theoretic Analysis of Hierarchical Protein Materials and Social Networks.” PLoS ONE 6(9): e23911. doi:10.1371/journal.pone.0023911

\bibitem[SK]{SK} Spivak, D.I., Kent, R.E. (2012) “Ologs: A Categorical Framework for Knowledge Representation.” {\em PLoS ONE} 7(1): e24274. doi:10.1371/journal.pone.0024274.

\bibitem[WeS]{WeS} Weinberger, S. (2011) “What is... Persistent Homology?” AMS.

\bibitem[WeA]{WeA} Weinstein, A. (1996) “Groupoids: unifying internal and external symmetry. {\em Notices of the AMS} Vol 43, no. 7, pp. 744 -- 752.

\bibitem[Wik]{Wik} \href{http://www.wikipedia.org}{\text Wikipedia} (multiple authors). Various articles, all linked with a hyperreference are scattered throughout this text. All accessed December 6, 2012 -- \today.

\bibitem[SpHome]{SpHome} \href{http://math.mit.edu/~dspivak/}{\text David Spivak's Homepage}.

\bibitem[CDSite]{CDSite} \href{http://categoricaldata.net/}{\text Categorical Data} website.

\end{thebibliography}

\end{document}
